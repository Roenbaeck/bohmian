%%%%%%%%%%%%%%%%%%%%%%%%%%%%%%%%%%%%%%%%%
% Journal Article
% LaTeX Template
% Version 1.4 (15/5/16)
%
% This template has been downloaded from:
% http://www.LaTeXTemplates.com
%
% Original author:
% Frits Wenneker[](http://www.howtotex.com) with extensive modifications by
% Vel (vel@LaTeXTemplates.com)
%
% License:
% CC BY-NC-SA 3.0[](http://creativecommons.org/licenses/by-nc-sa/3.0/)
%
%%%%%%%%%%%%%%%%%%%%%%%%%%%%%%%%%%%%%%%%%

\documentclass[twoside,twocolumn]{article}

% --- Add necessary math packages ---
\usepackage{amsmath}
\usepackage{amssymb}
\usepackage{amsthm}
\usepackage[sc]{mathpazo}
\usepackage[T1]{fontenc}
\linespread{1.05}
\usepackage{microtype}
\usepackage{bm}
\usepackage{tikz}
\usetikzlibrary{positioning, shapes.geometric}
\usepackage[english]{babel}
\usepackage[hmarginratio=1:1,top=32mm,columnsep=20pt]{geometry}
\usepackage[hang, small,labelfont=bf,up,textfont=it,up]{caption}
\usepackage{booktabs}
\usepackage{enumitem}
\setlist[itemize]{noitemsep}
\usepackage{abstract}
\renewcommand{\abstractnamefont}{\normalfont\bfseries}
\renewcommand{\abstracttextfont}{\normalfont\small\itshape}
\usepackage{titlesec}
\usepackage{fancyhdr}
\pagestyle{fancy}
\fancyhead{}
\fancyfoot{}
\fancyhead[C]{Seam-Driven Generative Geometry and Topology, Draft: \today}
\fancyfoot[RO,LE]{\thepage}
\usepackage{titling}
\usepackage{hyperref}
% amsthm already loaded above
\usepackage{bookmark}

% --- CONVENTIONAL NUMBERING SETUP ---
\renewcommand{\thesection}{\Roman{section}}
\titleformat{\section}[block]{\large\scshape\centering}{\thesection.}{1em}{}

\renewcommand{\thesubsection}{\Alph{subsection}}
\titleformat{\subsection}[block]{\bfseries\large}{\thesection.\thesubsection}{1em}{}

\renewcommand{\thesubsubsection}{\arabic{subsubsection}}
\titleformat{\subsubsection}[runin]{\normalfont\bfseries}{\thesection.\thesubsection.\thesubsubsection.}{0.5em}{}[\quad]

\newtheorem{definition}{Definition}[subsection]
\renewcommand{\thedefinition}{\thesection.\thesubsection.\arabic{definition}}

\newtheorem{requirement}{Requirement}[definition]
\renewcommand{\therequirement}{\thedefinition.\arabic{requirement}}

\newtheorem{proposition}{Proposition}[section]
\newtheorem{theorem}{Theorem}[section]
\theoremstyle{remark}
\newtheorem{remark}{Remark}[section]
\theoremstyle{plain}
\theoremstyle{remark}
\newtheorem{remark}{Remark}[section]
\theoremstyle{plain}

\renewcommand{\theequation}{\thesection.\arabic{equation}}
\numberwithin{equation}{section}

%----------------------------------------------------------------------------------------
%	TITLE SECTION
%----------------------------------------------------------------------------------------

\setlength{\droptitle}{-4\baselineskip}

\pretitle{\begin{center}\Huge\bfseries}
\posttitle{\end{center}}
\title{\vspace{5mm}\fontsize{24pt}{10pt}\selectfont\textbf{Seam-Driven Generative Geometry and Topology}}
\author{%
\\[2mm] %
\textsc{Lars Rönnbäck}\thanks{Content iteratively generated by Gemini 2.5 Pro and references added/expanded by Claude 3.7 Sonnet, based on ideas by Lars Rönnbäck.}
\\[1ex] %
\normalsize Stockholm University \\
\normalsize \href{mailto:lars.ronnback@anchormodeling.com}{lars@uptochange.com}
\\[1ex] %
}
\date{}
\renewcommand{\maketitlehookd}{%
\begin{abstract}
\noindent This paper presents \emph{seam-driven generative geometry and topology}, a scalar-first organizational framework in which geometric and topological structures on a base space \(U\) are produced from a scalar function \(s: U \to \mathbb{R}\) (the \emph{seam}) together with an explicit \emph{Rule} \(\mathcal{R}\). The triplet \((U, s, \mathcal{R})\) recovers a range of standard constructions---pseudo-metrics, (pseudo-)Riemannian tensors, handle decompositions, warped products, and optimal-transport geometries---within a single notation. We formalize minimal requirements for valid Rules, introduce a few seam-centric variants and combinations, and show how several classical results can be rephrased in this language. The goal is synthesis and a constructive point of view rather than a claim of novelty for each individual construction.
\end{abstract}
}

%----------------------------------------------------------------------------------------

\begin{document}

\maketitle
\thispagestyle{fancy}

%----------------------------------------------------------------------------------------
%	ARTICLE CONTENTS
%----------------------------------------------------------------------------------------

\section{Introduction}

Standard presentations of geometry begin with a metric or a manifold structure postulated \emph{a priori}. Here we emphasize a complementary viewpoint: start with a base space \(U\) and a scalar field \(s\) (the \emph{seam}), then let an explicit Rule \(\mathcal{R}\) produce geometric data. The resulting triplet \((U, s, \mathcal{R})\) packages a number of familiar constructions in a uniform way: quadratic seams recover Euclidean and Minkowski spaces; logarithmic seams reproduce standard conformal factors for spherical/hyperbolic models; Morse seams organize topology via handle attachment; and warped-product seams recover standard spherically symmetric ans\"atze used in general relativity.

This scalar-first perspective offers:
\begin{itemize}
    \item \textbf{Generativity and Constructivity:} Complex geometries and topologies arise from elementary seams under fixed Rules.
    \item \textbf{Synthesis (organizational):} Riemannian, pseudo-Riemannian, conformal, Kähler, discrete, and Wasserstein constructions can be discussed within one scalar-first notation; the aim is a repackaging/synthesis rather than a claim of a new unifying theory.
    \item \textbf{Bridge to Topology:} Morse-Seam Rules simultaneously produce CW-complexes and local Riemannian patches.
    \item \textbf{Applications (illustrative):} Connections to physics- and data-science-motivated constructions (e.g.\ standard warped-product ans\"atze and optimal-transport geometry) are included as examples.
\end{itemize}

We formalize the framework, expand the repertoire of Rules, and reprove several classical statements in this seam-centric language. We also include a few seam-based density/identifiability observations and a discrete-to-continuum approximation result for mesh-based distances. Proofs left as sketches are marked \textbf{[Sketch --- to be replaced with full rigorous version later]} for easy future expansion.

\section{Framework: Seams and Rules}

Let \(U\) be a paracompact Hausdorff space (or discrete set) equipped with local structures \(\mathcal{T}_U\) (differentiable atlas, graph adjacency, cell complex, measure, etc.).

\begin{definition}[Seam]
A \emph{seam} is a function \(s: U \to \mathbb{R}\) belonging to an admissible class \(\mathcal{S}(U)\) (e.g., \(C^\infty\), Morse, convex, Kantorovich potential). The seam encodes local tension, scale, height, or gluing data.
\end{definition}

\begin{definition}[Rule]\label{def:rule}
A \emph{Rule} is a natural, local assignment that takes admissible seam data \((U,s,\mathcal{T}_U)\) to a geometric output \(G\) (pseudo-metric \(D\), tensor \(g\), CW-complex, foliation, Wasserstein space, etc.).

In practice, we impose the following minimal axioms (tailored to the chosen output type):
\begin{enumerate}
    \item \textbf{Locality (restriction).} For every open \(V\subseteq U\), the output on \(V\) depends only on \(s|_V\) and \(\mathcal{T}_U|_V\).
    \item \textbf{Gluing (sheaf condition).} If \(\{V_\alpha\}\) covers \(U\) and the locally generated outputs \(G_\alpha=\mathcal{R}(V_\alpha,s|_{V_\alpha})\) agree on overlaps, then there is a unique global object \(G\) with \(G|_{V_\alpha}=G_\alpha\).
    \item \textbf{Functoriality (coordinate invariance).} For every structure-preserving map \(\varphi:U\to U'\) in the relevant category (e.g.\ diffeomorphism), one has \(\mathcal{R}(U,\varphi^\ast s')=\varphi^\ast\mathcal{R}(U',s')\).
    \item \textbf{Scale covariance (optional, rule-specific).} When \(s\) enters as a scale field, we may require an explicit covariance such as \(\mathcal{R}(U,s+c)=e^{2c}\mathcal{R}(U,s)\) (conformal-type). We do \emph{not} impose a blanket pointwise ``monotonicity in \(s\)'' axiom, since many useful rules (e.g.\ gradient-based ones) are invariant under \(s\mapsto s+c\) and depend on derivatives of \(s\).
\end{enumerate}
\end{definition}

\begin{requirement}[Minimal for Pseudo-Metric Rules]
When \(G = D\), the generated distance must satisfy non-negativity, identity of indiscernibles (pseudo), symmetry, and triangle inequality.
\label{req:RuleReqs}
\end{requirement}

\begin{theorem}[Local seam models and limitations (Informal)]\texorpdfstring{}{ }
Many geometric structures can be generated locally from scalar data, but the expressive power depends strongly on the chosen Rule. For example:
\begin{itemize}
    \item Hessian-type Rules generate \emph{Hessian metrics}, a special subclass of (pseudo-)Riemannian metrics (e.g.\ those arising in information geometry); not every metric is locally Hessian.
    \item Conformal and warped-product Rules recover large standard families of metrics once a background metric and splitting are fixed.
    \item Morse seams reconstruct the topology of compact manifolds via sublevel sets, and can be combined with local metric patches produced by other Rules.
\end{itemize}
\noindent In particular, statements of the form ``every smooth (pseudo-)Riemannian metric is (locally) a Hessian'' are false without additional structure/conditions; adding ``up to conformal/warped factors'' can make a statement formally true but too weak to be informative.
\textbf{[Sketch --- to be replaced with a precise version later; the point is to delimit what is recovered by each Rule rather than to claim that every metric is of a single special form.]}
\end{theorem}

\section{Rules and Generated Structures}

\subsection{Hessian, Conformal, and Gradient Rules}
In this section, we will explore specific Rules $\mathcal{R}$ and examine the geometries $(U, D)$ they generate for various choices of $U$ and $s$. We focus first on rules applicable when $U$ possesses a differentiable structure.

\subsection{The Hessian Rule ($\mathcal{R}_{\text{Hessian}}$)}
% Reset definition counter for this subsection
\setcounter{definition}{0}

Perhaps the most direct way to generate a tensor structure from a scalar field in a differentiable setting is by considering its second derivatives.

\begin{definition}[Hessian Rule]
Let $U$ be a differentiable manifold equipped with local coordinates $\{x^i\}$. Let the class of admissible seams $\mathcal{S}(U)$ be the set of twice continuously differentiable functions ($C^2$) $s: U \to \mathbb{R}$. The \emph{Hessian Rule}, denoted $\mathcal{R}_{\text{Hessian}}$, generates a symmetric tensor field $g$ of type (0,2) with components in local coordinates given by:
\begin{equation}
g_{ij}(u) = \frac{\partial^2 s}{\partial x^i \partial x^j}(u)
\label{eq:hessian_rule}
\end{equation}
If this tensor $g$ is non-degenerate (i.e., $\det(g) \neq 0$) and has a constant signature, it defines a pseudo-Riemannian metric on $U$. If $g$ is also positive definite ($\det(g) > 0$ and appropriate signature), it defines a Riemannian metric. The distance function $D = \mathcal{R}_{\text{Hessian}}(s; U)$ is then defined as the standard geodesic distance associated with the metric $g$.
\end{definition}

\subsection{A jet-decoder meta-rule (proposal)}
\label{subsec:jet-decoder}

One natural question is whether the collection of Rules can itself be derived from a single ``meta-Rule'' that reads progressively richer local data of a seam \(s\).
Motivated by this, one may consider a \emph{universal jet-decoder} Rule \(\mathcal{R}_\star\) that takes as input the low-order jets of \(s\) at each point (e.g.\ \(0\)-, \(1\)-, and \(2\)-jet), producing in stratified layers:
\begin{itemize}
    \item a Morse filtration/topological decomposition from the \(0\)-jet (values of \(s\)),
    \item a foliation/transverse scaling structure on regular regions from the \(1\)-jet (\(\nabla s\)),
    \item local Hessian patches near critical points or along leaves from the \(2\)-jet (\(\mathrm{Hess}\,s\)),
\end{itemize}
with a global gluing/completion step specified by a canonical prescription (for example, a variational principle constrained by the jet data).

\paragraph{Assessment (informal).}
As stated, \(\mathcal{R}_\star\) is compelling as an organizational device, but several claims need qualification to avoid overclaiming:
\begin{enumerate}
    \item \textbf{Topology from \(0\)-jets is conditional.} Morse/handle conclusions require standard hypotheses (e.g.\ \(s\) Morse and suitable compactness/properness assumptions); otherwise the filtration can fail to encode topology cleanly.
    \item \textbf{\(1\)-jet data does not canonically fix a metric.} While \(\nabla s\neq 0\) induces a foliation on regular regions, turning \(|\nabla s|\) into a \emph{canonical} transverse scale is not intrinsic without a background metric or extra prescription.
    \item \textbf{\(2\)-jet patches do not make metrics dense by themselves.} Hessian metrics are a proper subclass; adding conformal/warped degrees of freedom can enlarge expressivity, but ``density in \(\mathrm{Met}(U)\) from a single seam'' needs a precise topology and a proof that the completion step is not doing all the work.
    \item \textbf{Canonical gluing is the hard part.} Partitions of unity are not canonical, and conformal/warped transitions generally require choices; a variational completion can be well-posed, but then \(\mathcal{R}_\star\) becomes a PDE-defined Rule whose uniqueness and functoriality must be stated carefully.
    \item \textbf{Discrete/OT exactness is unlikely as stated.} ``Recover exactly'' is too strong in general; what is more plausible is a discretization scheme that yields consistent approximations under appropriate hypotheses.
\end{enumerate}
Overall, \(\mathcal{R}_\star\) looks most defensible as a \emph{meta-framework} (``choose jet layers and a completion functional'') rather than a single canonical Rule that eliminates external choices.
% (internal note removed) 
The use of Hessian structures to define metrics has been extensively studied in information geometry \cite{Amari2016,Shima2007}.

\noindent Note that this rule does not automatically guarantee a positive definite (Riemannian) metric or even a non-degenerate one; the properties of the generated tensor $g$ depend entirely on the choice of the seam function $s$. If $g$ is degenerate or changes signature, the resulting geometry may be non-standard or only defined piecewise. However, $g$ always defines a symmetric bilinear form on each tangent space, and the framework can still be explored. Let's examine some outcomes.

\subsubsection{Example: Euclidean Space}
Let $U = \mathbb{R}^n$ with standard Cartesian coordinates $(x^1, \dots, x^n)$. Consider the quadratic seam:
$$ s(x^1, \dots, x^n) = \frac{1}{2} \sum_{i=1}^n (x^i)^2 $$
Applying the Hessian rule \eqref{eq:hessian_rule}:
$$ g_{ij} = \frac{\partial^2}{\partial x^i \partial x^j} \left( \frac{1}{2} \sum_{k=1}^n (x^k)^2 \right) = \delta_{ij} $$
This is the standard Euclidean metric tensor. The generated geometry $(U, D)$ is $n$-dimensional Euclidean space $\mathbb{E}^n$.

\subsubsection{Example: Minkowski Spacetime}
Let $U = \mathbb{R}^4$ with coordinates $(x^0, x^1, x^2, x^3)$. Consider the seam representing the squared interval:
\begin{align}
s(x^0, x^1, x^2, x^3) = \notag \\ \frac{1}{2} \left[ (x^0)^2 - (x^1)^2 - (x^2)^2 - (x^3)^2 \right]
\end{align}
Applying the Hessian rule \eqref{eq:hessian_rule}, using the $(+,-,-,-)$ signature convention:
$$ g_{\mu\nu} = \eta_{\mu\nu} $$
This yields the Minkowski metric tensor. The generated geometry is 4-dimensional Minkowski spacetime.

\subsubsection{Example: Flat 2D Lorentzian Space}
Let $U = \mathbb{R}^2$ with coordinates $(x, y)$. Consider the seam $s(x, y) = xy$. Applying the Hessian rule \eqref{eq:hessian_rule}:
$$ g = \begin{pmatrix} 0 & 1 \\ 1 & 0 \end{pmatrix} $$
This corresponds to a flat 2D Lorentzian geometry.

\subsubsection{Example: Curved Riemannian Geometry}
Let $U = \mathbb{R}^2$ with coordinates $(x, y)$. Consider the seam $s(x, y) = \frac{1}{4} (x^2 + y^2)^2$. The Hessian rule yields the metric:
$$ g = \begin{pmatrix} 3x^2+y^2 & 2xy \\ 2xy & x^2+3y^2 \end{pmatrix} $$
This metric is positive definite for $(x,y) \neq (0,0)$ and indicates a curved Riemannian geometry.

\subsubsection{Example: Spherically Symmetric Geometries}
Let $U = \mathbb{R}^3$ and consider a spherically symmetric seam $s = f(r)$, where $r = \sqrt{x^2+y^2+z^2}$. The Hessian rule generates the metric:
$$ ds^2 = f''(r) dr^2 + \frac{f'(r)}{r} (r^2 d\theta^2 + r^2 \sin^2\theta d\phi^2) $$
For this to match the standard form $ds^2 = A(r) dr^2 + r^2 B(r) (d\Omega^2)$, we need $A(r) = f''(r)$ and $B(r) = f'(r)/r$. This implies $B'(r) = (r f''(r) - f'(r))/r^2$, so $r B'(r) = f''(r) - f'(r)/r = A(r) - B(r)$, or $A(r) = B(r) + r B'(r)$. This is a constraint satisfied by flat space ($A=1, B=1 \implies s=r^2/2$) but not generally by other important metrics like Schwarzschild.

\subsubsection{Example: Changing Signature}
Let $U = \mathbb{R}^2$ with coordinates $(x, y)$. Consider the seam $s(x, y) = \cos(x) + \frac{1}{2} y^2$. The Hessian rule yields $g = \begin{pmatrix} -\cos(x) & 0 \\ 0 & 1 \end{pmatrix}$. The geometry changes type depending on the sign of $\cos(x)$.

\subsubsection{Summary for Hessian Rule}
The Hessian Rule $\mathcal{R}_{\text{Hessian}}$ provides a direct link from a scalar potential $s$ ($C^2$) to a metric tensor $g$. It naturally generates flat Euclidean and Minkowski spaces. It can generate curved geometries, both Riemannian and pseudo-Riemannian. However, it does not guarantee positive definiteness and seems unable to generate certain important geometries like those with constant non-zero curvature directly on $\mathbb{R}^n$. It defines a specific class of geometries whose metric tensor is the Hessian of a potential.

\subsection{The Conformal Rule ($\mathcal{R}_{\text{Conf}}$)}
% Reset definition counter for this subsection
\setcounter{definition}{0}

Another natural way to relate a scalar field $s$ to a metric $g$ is by using $s$ to define a local scaling, or conformal factor, relative to some pre-existing background metric on $U$.

\begin{definition}[Conformal Rule]
Let $U$ be a differentiable manifold equipped with a background pseudo-Riemannian metric $h_{ij}$. Typically, for $U=\mathbb{R}^n$, $h_{ij}$ is taken to be the standard Euclidean metric $\delta_{ij}$. (It should be noted that this background metric $h_{ij}$ is conceptually flexible; it could potentially be derived from another seam $s_0$ using a different rule, such as $\mathcal{R}_{\text{Hessian}}$.) Let the class of admissible seams $\mathcal{S}(U)$ be the set of sufficiently smooth (e.g., continuous or differentiable) functions $s: U \to \mathbb{R}$. The \emph{Conformal Rule}, denoted $\mathcal{R}_{\text{Conf}}$, generates a pseudo-Riemannian metric tensor $g$ given by:
\begin{equation}
g_{ij}(u) = e^{2s(u)} h_{ij}(u)
\label{eq:conformal_rule}
\end{equation}
The distance function $D = \mathcal{R}_{\text{Conf}}(s; U)$ is then defined as the standard geodesic distance associated with the metric $g$.
\end{definition}
% ADDED CITATION BASED ON GUIDE:
Conformal transformations and their geometric properties are well-established in differential geometry \cite{Petersen2006,Lee2018}.

\noindent In this rule, the seam $s$ directly controls the local "stretching" factor $e^s$ applied to the background metric $h$. If $h$ is positive definite (Riemannian), then $g$ will also be positive definite. The resulting geometry $(U, g)$ is conformally equivalent to the background geometry $(U, h)$.

% ... rest of Conformal Rule subsection (Examples III.B.i - III.B.v + Summary) ...
\subsubsection{Example: Euclidean Space}
Let $U = \mathbb{R}^n$ with $h_{ij} = \delta_{ij}$. To generate $g_{ij} = \delta_{ij}$, we need $e^{2s} = 1$, implying the trivial seam $s=0$.

\subsubsection{Example: Spherical Geometry}
Let $U = \mathbb{R}^2$ with $h_{ij} = \delta_{ij}$. The metric of a sphere of radius $R$ in stereographic coordinates $ds^2 = \frac{4R^4}{(R^2 + x^2 + y^2)^2}(dx^2 + dy^2)$ is generated by the seam:
$$ s(x, y) = \ln(2R^2) - \ln(R^2 + x^2 + y^2) $$
via the Conformal Rule.

\subsubsection{Example: Hyperbolic Geometry (Upper Half-Plane)}
Let $U = \{ (x, y) \in \mathbb{R}^2 \mid y > 0 \}$ with $h_{ij} = \delta_{ij}$. The hyperbolic metric $ds^2 = \frac{R^2}{y^2} (dx^2 + dy^2)$ is generated by the seam:
$$ s(x, y) = \ln(R) - \ln(y) $$
via the Conformal Rule.

\subsubsection{Example: Flat, Conformally Distorted Plane}
Let $U = \mathbb{R}^2$ with $h_{ij} = \delta_{ij}$. Consider $s(x, y) = xy$. The Conformal Rule yields:
$$ g = \begin{pmatrix} e^{2xy} & 0 \\ 0 & e^{2xy} \end{pmatrix} $$
This metric is intrinsically flat ($K=0$) but not globally Euclidean.

\subsubsection{Summary for Conformal Rule}
The Conformal Rule $\mathcal{R}_{\text{Conf}}$ interprets $s$ as controlling the logarithm of a local scale factor applied to a background metric $h$. Its strengths are generating conformally flat geometries like spheres and hyperbolic spaces (when $h=\delta$), and preserving metric type. Its limitation is that it can only generate geometries conformally equivalent to the background. (Notably, the background $h$ itself could be the result of applying another rule, like $\mathcal{R}_{\text{Hessian}}$, to a different seam, allowing for the generation of metrics conformal to non-flat Hessian geometries.) It cannot change metric signature (e.g., Euclidean to Lorentzian) with real $s$.

% --- NEW SECTION FOR GRADIENT RULE ---
\subsection{The Gradient Rule (\( \mathcal{R}_{\text{Grad}} \))}
% Reset definition counter for this subsection
\setcounter{definition}{0}

This rule uses the magnitude of the gradient of the seam function $s$ to define an isotropic scaling factor relative to a background metric.

\begin{definition}[Gradient Rule]
Let \( U \) be a differentiable manifold with local coordinates \( \{x^i\} \) and a background Euclidean metric \( h_{ij} = \delta_{ij} \) in these coordinates. Let \( \mathcal{S}(U) \) be the set of \( C^1 \) functions \( s: U \to \mathbb{R} \). The \emph{Gradient Rule}, \( \mathcal{R}_{\text{Grad}} \), defines a metric tensor \( g \) with components:
\begin{equation}
g_{ij}(u) = |\nabla s(u)|^2 \delta_{ij}
\label{eq:gradient_rule}
\end{equation}
where \( |\nabla s|^2 = \sum_{k=1}^n (\partial s / \partial x^k)^2 \) is the squared magnitude of the gradient of \( s \) with respect to the background Euclidean metric. If \( \nabla s(u) \neq 0 \) for all \( u \) in a region, then \( g \) is a Riemannian metric in that region, and \( D = \mathcal{R}_{\text{Grad}}(s; U) \) is the geodesic distance under \( g \).
\end{definition}
% ADDED CITATION BASED ON GUIDE:
The interpretation of gradient magnitude as a geometric scaling factor connects to level set methods and eikonal equations \cite{Sethian1999}.

This Rule interprets \( s \) as a potential whose gradient’s magnitude directly scales an isotropic metric (conformally Euclidean). It requires only first derivatives of \( s \), unlike the Hessian Rule which requires second derivatives.

% ... rest of Gradient Rule subsection (Examples III.C.i - III.C.v + Summary) ...
\subsubsection{Example: Euclidean Space}
Let $U = \mathbb{R}^n$ with coordinates $(x^1, \dots, x^n)$ and background $\delta_{ij}$. Consider the linear seam:
$$ s(x^1, \dots, x^n) = x^1 $$
The gradient is $\nabla s = (1, 0, \dots, 0)$. The squared magnitude is $|\nabla s|^2 = 1^2 = 1$. Applying the Gradient Rule \eqref{eq:gradient_rule}:
$$ g_{ij} = (1) \delta_{ij} = \delta_{ij} $$
This generates the standard Euclidean metric. Any seam $s = \sum a_k x^k + c$ with $\sum a_k^2 = 1$ would also yield the Euclidean metric.

\subsubsection{Example: Radially Scaled Space}
Let $U = \mathbb{R}^n$ with coordinates $(x^1, \dots, x^n)$ and background $\delta_{ij}$. Consider the quadratic seam:
$$ s(x^1, \dots, x^n) = \frac{1}{2} \sum_{k=1}^n (x^k)^2 = \frac{1}{2} r^2 $$
The gradient is $\nabla s = (x^1, x^2, \dots, x^n)$. The squared magnitude is $|\nabla s|^2 = \sum_{k=1}^n (x^k)^2 = r^2$. Applying the Gradient Rule:
$$ g_{ij} = r^2 \delta_{ij} $$
This generates a metric $ds^2 = r^2 ( (dx^1)^2 + \dots + (dx^n)^2 ) = r^2 d\mathbf{x}^2$. This is a conformally flat metric. It is Riemannian for $r \neq 0$ but becomes degenerate ($g_{ij}=0$) at the origin $r=0$, where $\nabla s = 0$. Geodesics in this space are related to circles passing through the origin in the underlying Euclidean space.

\subsubsection{Example: Constant Seam}
Let $U = \mathbb{R}^n$. Consider a constant seam $s(u) = c$.
The gradient is $\nabla s = (0, \dots, 0)$. The squared magnitude is $|\nabla s|^2 = 0$. Applying the Gradient Rule:
$$ g_{ij} = (0) \delta_{ij} = 0 $$
This generates a completely degenerate tensor $g=0$. The associated distance $D(u, v)$ would be 0 for all $u, v$, violating the metric property unless $U$ is a single point. This highlights the importance of the condition $\nabla s \neq 0$.

\subsubsection{Relation to Conformal Rule}
The Gradient Rule $g_{ij} = |\nabla s|^2 \delta_{ij}$ always produces a metric conformally related to the background Euclidean metric $\delta_{ij}$. Comparing with the Conformal Rule $g_{ij} = e^{2\tilde{s}} \delta_{ij}$, we see that the Gradient Rule generates the same geometry as the Conformal Rule if we choose the conformal seam $\tilde{s}$ such that:
$$ e^{2\tilde{s}} = |\nabla s|^2 $$
This requires $|\nabla s|^2 > 0$, and gives $\tilde{s} = \ln(|\nabla s|)$. Therefore, any geometry generated by $\mathcal{R}_{\text{Grad}}$ (where $\nabla s \neq 0$) can also be generated by $\mathcal{R}_{\text{Conf}}$. However, the converse is not true: not every positive conformal factor $F(u) = e^{2\tilde{s}(u)}$ can be expressed as the squared magnitude of a gradient $|\nabla s(u)|^2$ for some $C^1$ function $s$. For example, the spherical metric factor $F = 4R^4 / (R^2+r^2)^2$ cannot be written as $|\nabla s|^2$ globally on $\mathbb{R}^2$. Thus, $\mathcal{R}_{\text{Grad}}$ generates a specific subset of conformally Euclidean geometries.

\subsubsection{Summary for Gradient Rule}
The Gradient Rule \( \mathcal{R}_{\text{Grad}} \) interprets the seam \( s \) via the magnitude of its gradient, requiring only \( C^1 \) smoothness.
\begin{itemize}
    \item It generates an isotropic metric \( g_{ij} = |\nabla s|^2 \delta_{ij} \), which is always conformally Euclidean.
    \item It can reproduce Euclidean space ($s=x^1$).
    \item It generates non-trivial conformally flat spaces (e.g., $s=r^2/2$ gives $g_{ij}=r^2 \delta_{ij}$).
    \item Requires $\nabla s \neq 0$ for the metric to be Riemannian (non-degenerate). Where $\nabla s = 0$, the geometry degenerates.
    \item Since $|\nabla s|^2 \ge 0$ and $\delta_{ij}$ is positive definite, it can only generate Riemannian or degenerate metrics, not pseudo-Riemannian metrics with mixed signatures (like Minkowski) from a Euclidean background.
    \item It generates a subset of the geometries accessible via the Conformal Rule $\mathcal{R}_{\text{Conf}}$, specifically those where the conformal factor can be written as $|\nabla s|^2$.
\end{itemize}
This rule offers an alternative mechanism based on first derivatives, leading naturally to isotropic scaling.


\subsection{Graph-Based Rules ($\mathcal{R}_{\text{Graph}}$)}
% Reset definition counter for this subsection
\setcounter{definition}{0}

The Hessian, Conformal, and Gradient rules rely fundamentally on the differentiable structure of the base space $U$. To extend the framework to spaces involving discrete components, such as lattices ($\mathbb{N}^n$) or mixed spaces ($\mathbb{N} \times \mathbb{R}$), a different approach is needed. Graph-based rules offer a natural pathway by interpreting $U$ as a set of vertices and using the seam $s$ to define connection costs (edge weights).

\begin{definition}[Graph Rule Framework]
Let $U$ be a base set, potentially composed of discrete and/or continuous components. A \emph{Graph-Based Rule}, $\mathcal{R}_{\text{Graph}}$, generates a distance function $D$ through the following conceptual steps:
\begin{enumerate}
    \item \textbf{Graph Structure:} Define an underlying graph structure $G = (U, E)$ on the base set $U$. This involves specifying the vertices (points in $U$) and edges $E$, representing allowed "connections" or adjacencies. For continuous or hybrid spaces, this might involve discretization or defining infinitesimal connections.
    \item \textbf{Weight Assignment:} Use the seam function $s: U \to \mathbb{R}$ to assign a non-negative weight $w(e)$ or cost to each edge $e \in E$. This is the core interpretive step for graph rules. For continuous spaces, this translates to defining a cost density or local metric element $ds$.
    \item \textbf{Distance Calculation:} Define the distance $D(u, v)$ between any two points $u, v \in U$ as the infimum of the total weight/cost along all possible paths connecting $u$ and $v$. For discrete graphs, this is the standard shortest path distance. For continuous/hybrid spaces, this involves integrating the cost density along paths.
\end{enumerate}
\end{definition}
% (internal note removed)
The discrete metric structures generated by graph rules have been studied extensively in spectral graph theory \cite{Chung1997} and discrete differential geometry \cite{Bobenko2015}.

\subsubsection{Candidate Rule 1: Cost from Seam Difference ($\mathcal{R}_{\text{Graph-}\Delta s}$)}
A simple rule for discrete graphs defines the weight of an edge $e=\{u, v\}$ based on the difference in seam values at its endpoints:
$$ w(e) = w(u, v) = |s(v) - s(u)| $$
If $s$ is constant, all edge weights are 0, leading to $D(u,v)=0$ within connected components. Variations could include adding a base cost: $w(u, v) = \epsilon + |s(v) - s(u)|$.

\subsubsection{Example: Weighted Lattice ($\mathbb{N} \times \mathbb{N}$)}
Let $U = \mathbb{N} \times \mathbb{N}$ with edges between neighbors $(i, j)$ and $(i', j')$ if $|i-i'|+|j-j'|=1$. Let $s(i, j) = i + j$. Then $w((i,j), (i+1,j)) = |(i+1+j) - (i+j)| = 1$, and $w((i,j), (i,j+1)) = |(i+j+1) - (i+j)| = 1$. This recovers the standard Manhattan distance $D((i,j), (k,l)) = |i-k| + |j-l|$. If $s(i, j) = (i+j)^2$, weights become non-uniform, e.g., $w((i,j), (i+1,j)) = |(i+1+j)^2 - (i+j)^2| = 2(i+j)+1$.

\subsubsection{Example: Hybrid Space ($\mathbb{N} \times \mathbb{R}$)}
Let $U = \mathbb{N} \times \mathbb{R}$. Define graph structure via adjacency: $(i, x)$ is connected to $(i+1, x)$ and $(i-1, x)$. Within layer $i$, points $(i, x)$ and $(i, y)$ are connected infinitesimally. Let $s(i, x) = i^2 + f(x)$. A path cost could combine discrete jump costs $w((i,x), (i+1,x)) = |s(i+1,x)-s(i,x)| = |(i+1)^2-i^2| = |2i+1|$ with continuous path integration using a local metric derived from $s$, e.g., $ds_{i} = \sqrt{(\partial s / \partial x)^2 dx^2} = |f'(x)|dx$. Defining the overall distance $D$ rigorously requires careful treatment of mixed path types.

\subsubsection{Candidate Rule 2: Conformal-Inspired Cost ($\mathcal{R}_{\text{Graph-Exp s}}$)}
Inspired by the Conformal Rule, we might define edge weights based on the average scale factor between nodes. For an edge $e=\{u, v\}$ and a base length $\epsilon_e$:
$$ w(e) = \epsilon_e \left( \frac{e^{s(u)} + e^{s(v)}}{2} \right) $$
Or in the continuous limit, a local metric $ds^2 = e^{2s(x)} dx^2$.

\subsubsection{Candidate Rule 3: Gradient-Magnitude Cost ($\mathcal{R}_{\text{Graph-}|\nabla s|}$)}
To approximate the continuous Gradient Rule \(g=|\nabla s|^2 g_0\) on a smooth manifold, one can define weights using the background edge length and the gradient magnitude of the seam. For an embedded edge \(e=\{u,v\}\) with background length \(\ell_0(u,v)\) (e.g.\ the \(g_0\)-geodesic distance between \(u\) and \(v\)), define
$$
w(e)=\ell_0(u,v)\left(\frac{|\nabla s|_{g_0}(u)+|\nabla s|_{g_0}(v)}{2}\right).
$$
In the continuum limit this is a Riemann-sum discretization of the length functional \(L_g(\gamma)=\int_\gamma |\nabla s|_{g_0}\,ds_{g_0}\), which is the geodesic length for \(g=|\nabla s|_{g_0}^2g_0\).

\subsubsection{Example: Standard Lattice ($\mathbb{N} \times \mathbb{N}$)}
Using $\mathcal{R}_{\text{Graph-Exp s}}$ with $s(i, j)=0$ and $\epsilon_e=1$ for all edges gives $w(e) = 1 (\frac{1+1}{2}) = 1$, recovering standard graph distance (Manhattan). If $s(i,j)$ is non-constant, the edge weights vary, creating a weighted graph.

\subsubsection{Summary for Graph Rules}
Graph-Based Rules provide a mechanism to generate geometry on non-differentiable or mixed spaces. The core challenge lies in defining the edge/connection structure $E$ and the weighting function $w(e; s)$ appropriately. Rule $\mathcal{R}_{\text{Graph-}\Delta s}$ uses seam differences, sensitive to gradients. Rule $\mathcal{R}_{\text{Graph-Exp s}}$ uses average seam exponentials, analogous to conformal scaling. These rules allow generating complex discrete or hybrid geometries from scalar seams but require careful formulation for consistency, especially for continuous limits or hybrid structures.

\subsection{Morse-Seam Rule \(\mathcal{R}_{\text{Morse}}\)}
Let \(s\) be Morse. Sublevel sets \(M_t = \{x \mid s(x) \le t\}\) build \(U\) by attaching handles at critical points. Locally near each critical point, apply \(\mathcal{R}_{\text{Hessian}}\) to quadratic approximations for Riemannian patches; glue on regular level sets via \(\mathcal{R}_{\text{Conf}}\) using \(s\) itself as transition factor.  
\textbf{Example:} \(s(x,y) = x^2 + y^2\) on \(\mathbb{R}^2\) generates a disk with standard radial metric. Generic Morse seam on \(S^2\) stitches the round sphere.

\subsection{Warped-Seam Rule \(\mathcal{R}_{\text{Warp}}\)}
On \(U = B \times F\), let radial seam \(s\) on base \(B\). Define
\[
g = g_B + \phi(s)^2 g_F, \quad
\phi(s) = e^s \text{ or } s \text{ or arbitrary positive function of } s.
\]
\textbf{Example:} FLRW cosmologies (\(s =\) cosmic time, \(\phi(s) =\) scale factor); de Sitter (\(\phi(s) = \cosh s\)); Schwarzschild isotropic coordinates via suitable radial seam.

\subsection{Optimal-Transport Seam Rule \(\mathcal{R}_{\text{OT}}\)}
Let \(s\) be a Kantorovich potential. Cost \(c(u,v) = s(u) + s^*(v)\) or \(|s(u)-s(v)|\) induces Wasserstein-\(p\) geometry on probability measures over \(U\).  
\textbf{Example:} Point-cloud data labeled by seam values yields intrinsic Wasserstein metric for clustering or generative modeling.

\section{Mathematical Properties of Rules}

In this section, we delve deeper into the mathematical properties of the geometries generated by the proposed rules, focusing on the conditions under which they produce valid (pseudo-)metric spaces and characterizing the specific classes of geometries they generate. We assume \( U \) possesses the necessary structure (e.g., differentiability, graph structure) required by each rule.

% Add theorem environments if not already defined in preamble
% Assuming amsthm is loaded
% The 'proof' environment is provided by amsthm

\subsection{Satisfaction of Pseudo-Metric Axioms}

A fundamental requirement (Requirement \ref{req:RuleReqs}) is that any valid rule \( \mathcal{R} \) must generate a distance function \( D = \mathcal{R}(s; U) \) that satisfies the pseudo-metric axioms (Non-negativity M1, Identity M2, Symmetry M3, Triangle Inequality M4).

\begin{proposition}[Metric Properties of Differentiable Rules] \label{prop:diff_metric_props}
Let \( \mathcal{R} \) be \( \mathcal{R}_{\text{Hessian}} \), \( \mathcal{R}_{\text{Conf}} \), or \( \mathcal{R}_{\text{Grad}} \), generating a tensor \( g \) from an admissible seam \( s \) on a differentiable manifold \( U \).
\begin{enumerate}
    \item If \( g \) is a Riemannian metric on a connected domain \( \Omega \subseteq U \), the associated geodesic distance \( D(u, v) = \inf \{\int \sqrt{g_{ij}\dot{\gamma}^i\dot{\gamma}^j} dt \mid \gamma: u \to v\} \) defines a true metric on \( \Omega \) \cite{Petersen2006, Lee2018}. % <-- UPDATED CITATION
    \item If \( g \) is pseudo-Riemannian, the geodesic distance (defined appropriately, e.g., for timelike or spacelike paths) satisfies M1-M3. The triangle inequality M4 holds under specific conditions related to causal structure and path types \cite{ONeill1983,BeemEhrlichEasley1996}. % <-- UPDATED CITATION
    Degeneracies (\( D(u,v)=0 \) for \( u \neq v \)) can occur, particularly for null-separated points.
\end{enumerate}
\end{proposition}
\begin{proof}[Proof Sketch]
(1) Follows from standard properties of Riemannian distance functions \cite{Petersen2006, Lee2018}. % <-- UPDATED CITATION
(2) Requires careful definition of distance in pseudo-Riemannian settings, often focusing on proper time/length along specific path types \cite{ONeill1983,BeemEhrlichEasley1996}. % <-- UPDATED CITATION
Degeneracy for null paths is inherent. Symmetry follows if the Lagrangian \( \sqrt{|g_{ij}\dot{\gamma}^i\dot{\gamma}^j|} \) is symmetric under time reversal.
\end{proof}

\begin{proposition}[Metric Properties of Graph Rules] \label{prop:graph_metric_props}
Let \( \mathcal{R}_{\text{Graph}} \) generate edge weights \( w(u, v) \) for adjacent nodes \( u, v \) in a graph \( G=(U, E) \) based on a seam \( s: U \to \mathbb{R} \). If the weight function satisfies \( w(u, v) \ge 0 \) (non-negativity) and \( w(u, v) = w(v, u) \) (symmetry) for all edges \( \{u, v\} \in E \), then the shortest path distance \( D(u, v) = \inf \{\sum_{e \in \gamma} w(e) \mid \gamma \text{ path from } u \text{ to } v\} \) defines a pseudo-metric on the connected components of \( G \). If \( w(u,v)>0 \) for all edges, \( D \) is a true metric on each component.
\end{proposition}
\begin{proof}
This is a standard result in graph theory \cite{Chung1997}. % <-- UPDATED CITATION
M1 and M2 are immediate. M3 follows from weight symmetry. M4 follows because concatenating optimal paths \( u \to v \) and \( v \to w \) yields a path \( u \to w \), whose length is an upper bound for the shortest path \( D(u,w) \). The specific rules proposed, \( w=|\Delta s| \) and \( w=(e^{s(u)}+e^{s(v)})/2 \), satisfy non-negativity and symmetry (assuming \( s \) is real).
\end{proof}

\noindent For hybrid spaces (\( \mathbb{N} \times \mathbb{R} \)), rigorously proving the metric properties requires a careful definition of the path integral and infimum over combined discrete jumps and continuous segments, which is beyond the scope of this initial presentation but constitutes an important direction for future work.

\subsection{Equivalence Classes of Seams}

Different seams may generate the same geometry under a given rule. We denote this equivalence by \( s_1 \sim_{\mathcal{R}} s_2 \).

\begin{proposition}[Seam Equivalence] \label{prop:seam_equiv}
Let \( U \) be \( \mathbb{R}^n \) with standard coordinates and \( h=\delta \) where applicable.
\begin{enumerate}
    \item \( \mathcal{R}_{\text{Hessian}} \): \( s_1 \sim_{\mathcal{R}_{\text{Hess}}} s_2 \iff s_1 - s_2 \) is an affine function, i.e., \( s_1(x) = s_2(x) + a \cdot x + b \) for some vector \( a \in \mathbb{R}^n \) and scalar \( b \in \mathbb{R} \).
    \item \( \mathcal{R}_{\text{Conf}} \): \( s_1 \sim_{\mathcal{R}_{\text{Conf}}} s_2 \iff s_1(u) = s_2(u) \) for all \( u \in U \).
    \item \( \mathcal{R}_{\text{Grad}} \): \( s_1 \sim_{\mathcal{R}_{\text{Grad}}} s_2 \iff |\nabla s_1(u)|^2 = |\nabla s_2(u)|^2 \) for all \( u \in U \). This holds if \( s_1 \) and \( s_2 \) are different solutions to the same Eikonal equation \( |\nabla s|^2 = F(u) \). (E.g., \( s_1=x^1 \) and \( s_2=x^2 \) both yield \( |\nabla s|^2=1 \) and \( g=\delta \)).
    \item \( \mathcal{R}_{\text{Graph-}\Delta s} \) (discrete graph): \( s_1 \sim_{\mathcal{R}_{\text{Graph-}\Delta s}} s_2 \iff s_1(u) = s_2(u) + c \) for some constant \( c \) (assuming identical adjacency and graph structure).
    \item \( \mathcal{R}_{\text{Graph-Exp s}} \) (discrete graph): \( s_1 \sim_{\mathcal{R}_{\text{Graph-Exp s}}} s_2 \iff s_1(u) = s_2(u) \) for all \( u \in U \).
\end{enumerate}
\end{proposition}
\begin{proof} % Proofs are brief, expand if needed
(1) \( \partial_i \partial_j s_1 = \partial_i \partial_j s_2 \iff \partial_i \partial_j (s_1 - s_2) = 0 \). Integrating twice yields \( s_1-s_2 \) is affine.
(2) \( e^{2s_1} h_{ij} = e^{2s_2} h_{ij} \implies e^{2s_1}=e^{2s_2} \implies s_1=s_2 \) (assuming \( h_{ij} \) is non-degenerate).
(3) \( |\nabla s_1|^2 \delta_{ij} = |\nabla s_2|^2 \delta_{ij} \implies |\nabla s_1|^2 = |\nabla s_2|^2 \).
(4) For an edge \( \{u, v\} \), \( |s_1(v)-s_1(u)| = |s_2(v)-s_2(u)| \). If \( s_1 = s_2 + c \), \( |(s_2(v)+c)-(s_2(u)+c)| = |s_2(v)-s_2(u)| \).
(5) For an edge \( \{u, v\} \), \( (e^{s_1(u)}+e^{s_1(v)})/2 = (e^{s_2(u)}+e^{s_2(v)})/2 \). For this to hold for all edges in a connected graph generally requires \( s_1(u) = s_2(u) \) for all \( u \).
\end{proof}

\begin{theorem}[Inverse seam (least-squares fit for the conformal graph rule)]\label{thm:inverse_seam}
Let \(G=(V,E)\) be a connected graph with background edge lengths \(\ell_0(u,v)>0\) for each edge \(\{u,v\}\in E\). Given target edge weights \(w^\ast(u,v)>0\), consider the least-squares energy over seams \(s:V\to\mathbb{R}\),
\[
\mathcal{E}(s)=\sum_{\{u,v\}\in E}\left(\ell_0(u,v)\frac{e^{s(u)}+e^{s(v)}}{2}-w^\ast(u,v)\right)^2.
\]
Under the change of variables \(X_u:=e^{s(u)}\in \mathbb{R}_{>0}\), the energy becomes a quadratic function
\[
\mathcal{E}(X)=\sum_{\{u,v\}\in E}\left(\frac{\ell_0(u,v)}{2}(X_u+X_v)-w^\ast(u,v)\right)^2,
\]
i.e.\ a convex quadratic program over the positive orthant. Moreover, if \(G\) is non-bipartite then the quadratic form is strictly convex on \(\mathbb{R}^{|V|}\), hence \(\mathcal{E}(X)\) has a unique global minimizer \(X^\ast\in \mathbb{R}^{|V|}\); if in addition \(X^\ast\in \mathbb{R}^{|V|}_{>0}\), then it corresponds to a unique minimizing seam \(s^\ast=\log X^\ast\).
\end{theorem}

\begin{proof}
Let \(\alpha_{uv}:=\ell_0(u,v)/2>0\). Writing the energy in terms of \(X\) gives
\[
\mathcal{E}(X)=\sum_{\{u,v\}\in E}\left(\alpha_{uv}(X_u+X_v)-w^\ast(u,v)\right)^2,
\]
which is a polynomial of degree two in \(X\). Hence \(\mathcal{E}(X)=X^\top H X-2c^\top X + K\) for some symmetric matrix \(H\), vector \(c\), and constant \(K\), and is therefore convex if and only if \(H\succeq 0\).

To identify \(H\), note that the quadratic part of a single edge term is
\[
\alpha_{uv}^2(X_u+X_v)^2=\alpha_{uv}^2(X_u^2+2X_uX_v+X_v^2).
\]
Summing over edges shows that \(H\) has diagonal entries \(H_{uu}=\sum_{v\sim u}\alpha_{uv}^2\) and off-diagonal entries \(H_{uv}=\alpha_{uv}^2\) when \(\{u,v\}\in E\) (and \(0\) otherwise), i.e.\ \(H\) is the weighted signless Laplacian up to a constant factor.

Equivalently, for any \(X\in\mathbb{R}^{|V|}\),
\[
X^\top H X=\sum_{\{u,v\}\in E}\alpha_{uv}^2(X_u+X_v)^2\ \ge\ 0,
\]
so \(H\succeq 0\). Moreover, \(X^\top H X=0\) if and only if \(X_u=-X_v\) for every edge \(\{u,v\}\), which is possible on a connected graph if and only if the graph is bipartite. Thus, if \(G\) is non-bipartite then \(H\succ 0\), so \(\mathcal{E}(X)\) is strictly convex on \(\mathbb{R}^{|V|}\) and has a unique global minimizer \(X^\ast\). If \(X^\ast>0\), then \(s^\ast=\log X^\ast\) is well-defined and is the unique minimizing seam.
\end{proof}

\begin{theorem}[Linearized discrete curvature under vertex scaling]\label{thm:discrete_curvature_linearized}
Let \((V,E,F)\) be a closed triangulated surface equipped with a piecewise-Euclidean background metric given by edge lengths \(\ell_0:E\to\mathbb{R}_{>0}\) satisfying the triangle inequalities on each face \(f\in F\). For \(u\in V\), let
\[
K_0(u):=2\pi-\sum_{f\ni u}\theta^{f}_u
\]
denote the discrete Gaussian curvature (angle defect), where \(\theta^{f}_u\) is the interior angle at \(u\) in the Euclidean triangle \(f\) with edge lengths \(\ell_0\).

For a vertex function \(s:V\to\mathbb{R}\), define a (discrete conformal) vertex scaling of edge lengths by
\[
\ell_s(u,v):=e^{\frac{s(u)+s(v)}{2}}\ell_0(u,v).
\]
Assume \(s\) is taken in a neighborhood of \(0\) small enough that \(\ell_s\) satisfies triangle inequalities on every face, so that the angle defects \(K_s(u)\) are well-defined. Let \(L^{\mathrm{cot}}\) be the cotangent Laplacian matrix of the background metric \(\ell_0\), i.e.\ for an interior edge \(\{u,v\}\in E\) adjacent to two faces with opposite angles \(\alpha_{uv},\beta_{uv}\),
\[
L^{\mathrm{cot}}_{uv}:=-(\cot\alpha_{uv}+\cot\beta_{uv}),\qquad
L^{\mathrm{cot}}_{uu}:=-\sum_{v\sim u}L^{\mathrm{cot}}_{uv}.
\]
Then the curvature map \(s\mapsto K_s\) is differentiable at \(s=0\), and its Jacobian is given by
\[
\left.\frac{\partial K_s(u)}{\partial s(v)}\right|_{s=0}
=-\frac{1}{2}L^{\mathrm{cot}}_{uv}.
\]
\end{theorem}

\begin{proof}[Proof sketch]
This is a standard result in discrete conformal geometry / discrete differential geometry: under the vertex scaling \(\ell_s(u,v)=e^{(s(u)+s(v))/2}\ell_0(u,v)\), the first variation of triangle angles can be expressed in terms of cotangents of the opposite angles, and assembling the variations around a vertex yields the cotangent Laplacian. Concretely, differentiating the angle defect
\(
K_s(u)=2\pi-\sum_{f\ni u}\theta^{f}_u
\)
reduces to summing the derivatives of \(\theta^{f}_u\) in each incident face. Using the law of cosines in each face and differentiating at \(s=0\), one obtains edge-local contributions proportional to \(\cot\) of the angles opposite \(\{u,v\}\), and the coefficient \(\frac12\) comes from the factor \(\frac{s(u)+s(v)}{2}\) in the exponent. Summing over the two faces adjacent to an interior edge \(\{u,v\}\) produces \((\cot\alpha_{uv}+\cot\beta_{uv})\) and hence the stated Jacobian \(-\frac12L^{\mathrm{cot}}\).
\end{proof}

\begin{remark}[Relation to the Conformal Graph Rule \(\mathcal{R}_{\text{Graph-Exp }s}\)]
In the graph rule used elsewhere in this paper one sets
\(
\ell_s(u,v)=\ell_0(u,v)\frac{e^{s(u)}+e^{s(v)}}{2}.
\)
At \(s=0\) this update agrees with the vertex scaling \(\ell_s(u,v)=e^{(s(u)+s(v))/2}\ell_0(u,v)\) to first order, since both have the same Taylor expansion \(\ell_s(u,v)=\ell_0(u,v)\bigl(1+\frac{s(u)+s(v)}{2}+O(\|s\|^2)\bigr)\). Thus, the curvature variation induced by \(\mathcal{R}_{\text{Graph-Exp }s}\) matches the cotangent Laplacian \emph{to first order at \(s=0\)}.
\end{remark}

\begin{theorem}[Linearized discrete curvature under vertex scaling]\label{thm:discrete_curvature_linearized}
Let \((V,E,F)\) be a closed triangulated surface equipped with a piecewise-Euclidean background metric given by edge lengths \(\ell_0:E\to\mathbb{R}_{>0}\) satisfying the triangle inequalities on each face \(f\in F\). For \(u\in V\), let
\[
K_0(u):=2\pi-\sum_{f\ni u}\theta^{f}_u
\]
denote the discrete Gaussian curvature (angle defect), where \(\theta^{f}_u\) is the interior angle at \(u\) in the Euclidean triangle \(f\) with edge lengths \(\ell_0\).

For a vertex function \(s:V\to\mathbb{R}\), define a (discrete conformal) vertex scaling of edge lengths by
\[
\ell_s(u,v):=e^{\frac{s(u)+s(v)}{2}}\ell_0(u,v).
\]
Assume \(s\) is taken in a neighborhood of \(0\) small enough that \(\ell_s\) satisfies triangle inequalities on every face, so that the angle defects \(K_s(u)\) are well-defined. Let \(L^{\mathrm{cot}}\) be the cotangent Laplacian matrix of the background metric \(\ell_0\), i.e.\ for an interior edge \(\{u,v\}\in E\) adjacent to two faces with opposite angles \(\alpha_{uv},\beta_{uv}\),
\[
L^{\mathrm{cot}}_{uv}:=-(\cot\alpha_{uv}+\cot\beta_{uv}),\qquad
L^{\mathrm{cot}}_{uu}:=-\sum_{v\sim u}L^{\mathrm{cot}}_{uv}.
\]
Then the curvature map \(s\mapsto K_s\) is differentiable at \(s=0\), and its Jacobian is given by
\[
\left.\frac{\partial K_s(u)}{\partial s(v)}\right|_{s=0}
=-\frac{1}{2}L^{\mathrm{cot}}_{uv}.
\]
\end{theorem}

\begin{proof}[Proof sketch]
This is a standard result in discrete conformal geometry / discrete differential geometry: under the vertex scaling \(\ell_s(u,v)=e^{(s(u)+s(v))/2}\ell_0(u,v)\), the first variation of triangle angles can be expressed in terms of cotangents of the opposite angles, and assembling the variations around a vertex yields the cotangent Laplacian. Concretely, differentiating the angle defect
\(
K_s(u)=2\pi-\sum_{f\ni u}\theta^{f}_u
\)
reduces to summing the derivatives of \(\theta^{f}_u\) in each incident face. Using the law of cosines in each face and differentiating at \(s=0\), one obtains edge-local contributions proportional to \(\cot\) of the angles opposite \(\{u,v\}\), and the coefficient \(\frac12\) comes from the factor \(\frac{s(u)+s(v)}{2}\) in the exponent. Summing over the two faces adjacent to an interior edge \(\{u,v\}\) produces \((\cot\alpha_{uv}+\cot\beta_{uv})\) and hence the stated Jacobian \(-\frac12L^{\mathrm{cot}}\).
\end{proof}

\begin{remark}[Relation to the Conformal Graph Rule \(\mathcal{R}_{\text{Graph-Exp }s}\)]
In the graph rule used elsewhere in this paper one sets
\(
\ell_s(u,v)=\ell_0(u,v)\frac{e^{s(u)}+e^{s(v)}}{2}.
\)
At \(s=0\) this update agrees with the vertex scaling \(\ell_s(u,v)=e^{(s(u)+s(v))/2}\ell_0(u,v)\) to first order, since both have the same Taylor expansion \(\ell_s(u,v)=\ell_0(u,v)\bigl(1+\frac{s(u)+s(v)}{2}+O(\|s\|^2)\bigr)\). Thus, the curvature variation induced by \(\mathcal{R}_{\text{Graph-Exp }s}\) matches the cotangent Laplacian \emph{to first order at \(s=0\)}.
\end{remark}

\begin{theorem}[Inverse seam (least-squares fit for the conformal graph rule)]\label{thm:inverse_seam}
Let \(G=(V,E)\) be a connected graph with background edge lengths \(\ell_0(u,v)>0\) for each edge \(\{u,v\}\in E\). Given target edge weights \(w^\ast(u,v)>0\), consider the least-squares energy over seams \(s:V\to\mathbb{R}\),
\[
\mathcal{E}(s)=\sum_{\{u,v\}\in E}\left(\ell_0(u,v)\frac{e^{s(u)}+e^{s(v)}}{2}-w^\ast(u,v)\right)^2.
\]
Under the change of variables \(X_u:=e^{s(u)}\in \mathbb{R}_{>0}\), the energy becomes a quadratic function
\[
\mathcal{E}(X)=\sum_{\{u,v\}\in E}\left(\frac{\ell_0(u,v)}{2}(X_u+X_v)-w^\ast(u,v)\right)^2,
\]
i.e.\ a convex quadratic program over the positive orthant. Moreover, if \(G\) is non-bipartite then the quadratic form is strictly convex on \(\mathbb{R}^{|V|}\), hence \(\mathcal{E}(X)\) has a unique global minimizer \(X^\ast\in \mathbb{R}^{|V|}\); if in addition \(X^\ast\in \mathbb{R}^{|V|}_{>0}\), then it corresponds to a unique minimizing seam \(s^\ast=\log X^\ast\).
\end{theorem}

\begin{proof}
Let \(\alpha_{uv}:=\ell_0(u,v)/2>0\). Writing the energy in terms of \(X\) gives
\[
\mathcal{E}(X)=\sum_{\{u,v\}\in E}\left(\alpha_{uv}(X_u+X_v)-w^\ast(u,v)\right)^2,
\]
which is a polynomial of degree two in \(X\). Hence \(\mathcal{E}(X)=X^\top H X-2c^\top X + K\) for some symmetric matrix \(H\), vector \(c\), and constant \(K\), and is therefore convex if and only if \(H\succeq 0\).

To identify \(H\), note that the quadratic part of a single edge term is
\[
\alpha_{uv}^2(X_u+X_v)^2=\alpha_{uv}^2(X_u^2+2X_uX_v+X_v^2).
\]
Summing over edges shows that \(H\) has diagonal entries \(H_{uu}=\sum_{v\sim u}\alpha_{uv}^2\) and off-diagonal entries \(H_{uv}=\alpha_{uv}^2\) when \(\{u,v\}\in E\) (and \(0\) otherwise), i.e.\ \(H\) is the weighted signless Laplacian up to a constant factor.

Equivalently, for any \(X\in\mathbb{R}^{|V|}\),
\[
X^\top H X=\sum_{\{u,v\}\in E}\alpha_{uv}^2(X_u+X_v)^2\ \ge\ 0,
\]
so \(H\succeq 0\). Moreover, \(X^\top H X=0\) if and only if \(X_u=-X_v\) for every edge \(\{u,v\}\), which is possible on a connected graph if and only if the graph is bipartite. Thus, if \(G\) is non-bipartite then \(H\succ 0\), so \(\mathcal{E}(X)\) is strictly convex on \(\mathbb{R}^{|V|}\) and has a unique global minimizer \(X^\ast\). If \(X^\ast>0\), then \(s^\ast=\log X^\ast\) is well-defined and is the unique minimizing seam.
\end{proof}

\subsection{Characterization of Generated Geometries}

We can characterize the specific classes of geometries generated by each rule.

\subsubsection{Hessian Rule Geometries}
The Hessian rule \( g_{ij} = \partial_i \partial_j s \) generates \emph{Hessian metrics}.
\begin{itemize}
    \item \textbf{Metric Type:} \( g \) is Riemannian if \( s \) is strictly convex, positive semi-definite if \( s \) is convex, and pseudo-Riemannian if the Hessian matrix \( (\partial_i \partial_j s) \) has the appropriate signature. Degeneracy occurs where \( \det(\text{Hess}(s)) = 0 \) \cite{Rockafellar1970}. % <-- UPDATED CITATION
    \item \textbf{Geometric Class:} These metrics are central to Information Geometry and Affine Differential Geometry \cite{Amari2016,Shima2007}. % <-- UPDATED CITATION
    Dually flat spaces in Information Geometry often possess metrics derived from the Hessian of a convex potential function (divergence) \cite{Amari2016}. % <-- Repeated citation okay here
    \item \textbf{Integrability:} A given tensor \( g_{ij} \) can be locally written as a Hessian, \( g_{ij}=\partial_i\partial_j s \), if and only if certain integrability conditions related to its curvature are met. For instance, if \( g \) is flat (\( R_{ijkl}=0 \)), it must be constant to be a Hessian globally on \( \mathbb{R}^n \). If \( g \) is derived from a K\"ahler potential \( s \) in complex geometry (\( g_{i\bar{j}} = \partial_i \partial_{\bar{j}} s \)), this imposes specific curvature properties \cite{Jost2017}. % <-- UPDATED CITATION (Using Jost as placeholder replacement)
\end{itemize}

\subsubsection{Conformal Rule Geometries}
The rule \( g_{ij} = e^{2s} h_{ij} \) generates geometries \emph{conformally equivalent} to the background \( (U, h) \).
\begin{itemize}
    \item \textbf{Metric Type:} The signature of \( g \) is the same as the signature of \( h \) (since \( e^{2s}>0 \)). Riemannian remains Riemannian.
    \item \textbf{Geometric Class:} If \( h = \delta \) is the flat Euclidean metric on \( U \subseteq \mathbb{R}^n \), then \( \mathcal{R}_{\text{Conf}} \) generates \emph{conformally flat} geometries. For \( n \ge 3 \), this is equivalent to the vanishing of the Weyl tensor \( W_{ijkl}=0 \). For \( n=2 \), all metrics are conformally flat \cite{DiFrancesco2012}. % <-- UPDATED CITATION (Using DiFrancesco as placeholder replacement)
    \item \textbf{Flexibility of Background:} While often assuming \( h=\delta \), the background \( h \) could itself be generated by another rule. If \( h_{ij} = (\mathcal{R}_{\text{Hessian}}(s_0))_{ij} = \partial_i \partial_j s_0 \), the resulting metric \( g_{ij} = e^{2s} (\partial_i \partial_j s_0) \) is conformal to a Hessian metric, potentially representing a richer geometric class dependent on two seams, \( s \) and \( s_0 \).
    \item \textbf{Curvature:} The curvature (e.g., scalar curvature \( \tilde{R} \) of \( g=e^{2s}h \)) is related to the curvature \( R \) of \( h \) and derivatives of \( s \) via known transformation laws, e.g., \( \tilde{R} = e^{-2s}(R - 2(n-1)\Delta_h s - (n-1)(n-2)|\nabla s|_h^2) \) for Riemannian \( h \) \cite{Besse1987}. % <-- UPDATED CITATION
\end{itemize}

\subsubsection{Gradient Rule Geometries}
The rule \( g_{ij} = |\nabla s|^2 \delta_{ij} \) (relative to a background \( \delta_{ij} \)) generates \emph{isotropic, conformally flat} geometries.
\begin{itemize}
    \item \textbf{Metric Type:} Since \( |\nabla s|^2 \ge 0 \), \( g \) is always Riemannian (positive definite) where \( \nabla s \neq 0 \) and degenerate (\( g=0 \)) where \( \nabla s = 0 \). It cannot generate pseudo-Riemannian geometries from a Riemannian background via a real seam \( s \).
    \item \textbf{Geometric Class:} These metrics are a subset of conformally flat metrics. The conformal factor \( F(u) = |\nabla s(u)|^2 \) must be the squared magnitude of a gradient. Not all positive functions \( F(u) \) can be written this way \cite{Sethian1999}. % <-- UPDATED CITATION
    Thus, \( \mathcal{R}_{\text{Grad}} \) is strictly less general than \( \mathcal{R}_{\text{Conf}} \) for generating conformally flat geometries. For example, the spherical metric factor \( F=4R^4/(R^2+r^2)^2 \) is not \( |\nabla s|^2 \) for any smooth \( s \) globally on \( \mathbb{R}^2 \).
    \item \textbf{Degeneracy:} Geodesics and distances are ill-defined at critical points of \( s \) where \( \nabla s = 0 \). The geometry collapses locally.
\end{itemize}

\subsubsection{Graph Rule Geometries}
The geometries generated by \( \mathcal{R}_{\text{Graph}} \) are weighted graphs where distances are shortest path lengths.
\begin{itemize}
    \item \textbf{Geometric Class:} These are discrete metric spaces. Their large-scale geometry depends heavily on the choice of \( s \) and the weighting rule.
    \item \textbf{Examples:} \( \mathcal{R}_{\text{Graph-Exp s}} \) with \( s=0 \) recovers standard graph distances (e.g., Manhattan distance on \( \mathbb{Z}^n \) with 2n-connectivity if edge lengths are 1). Non-zero \( s \) leads to non-homogeneous weighted graphs.
    \item \textbf{Continuum Limit:} Understanding the geometry generated by graph rules on increasingly fine discretizations of a manifold \( U \), and how it relates to the differential rules, requires careful analysis using tools like Gromov-Hausdorff convergence or discrete exterior calculus \cite{Ollivier2009,BobenkoSpringborn2007}. % <-- UPDATED CITATION
    This is an active research area.
    \item \textbf{Hybrid Spaces:} Defining and analyzing the geometry of hybrid spaces like \( \mathbb{N} \times \mathbb{R} \) under these rules requires rigorous formulation of path lengths combining discrete and continuous costs, ensuring the resulting distance satisfies metric properties, particularly the triangle inequality.
\end{itemize}

\subsection{Key Theorems}

\subsubsection{Reproofs via Seams (Simpler Scalar Arguments)}

\begin{theorem}[Gauss--Bonnet for Closed Oriented Surfaces via Seams]
Let \((M,g)\) be a closed oriented Riemannian surface. By the uniformization theorem there exists a Riemannian metric \(h\) of constant Gaussian curvature on \(M\) and a smooth seam \(s\in C^\infty(M)\) such that
\[
g = e^{2s} h = \mathcal{R}_{\text{Conf}}(s;h).
\]
Then
\[
\int_M K_g\,dA_g = 2\pi\chi(M),
\]
where \(K_g\) is the Gaussian curvature of \(g\) and \(\chi(M)\) is the Euler characteristic of \(M\).
\end{theorem}

\begin{proof}
The uniformization theorem guarantees that every conformal class on a closed orientable surface admits a unique (up to homothety) constant-curvature representative. Thus there exists a metric \(h\) with constant Gaussian curvature \(K_h\) (equal to \(+1\), \(0\), or \(-1\) after suitable rescaling, depending on \(\chi(M)\)) and a smooth seam \(s\) such that \(g\) is generated from \(h\) by the Conformal Rule:
\[
g = e^{2s} h.
\]

The Gaussian curvature transforms under this conformal change by the standard formula (see e.g.\ Petersen \cite{Petersen2006}, Prop.\ 8.3.3 or Lee \cite{Lee2018}, Thm.\ 3.5):
\[
K_g = e^{-2s}(K_h - \Delta_h s),
\]
where \(\Delta_h\) is the Laplace--Beltrami operator with respect to \(h\).

The Riemannian volume form scales as
\[
dA_g = e^{2s}\, dA_h.
\]

Multiplying these two relations yields
\[
K_g\, dA_g = (K_h - \Delta_h s)\, dA_h.
\]

Integrating over the closed manifold \(M\) (compact, without boundary) gives
\[
\int_M K_g\, dA_g = \int_M K_h\, dA_h - \int_M \Delta_h s\, dA_h.
\]

The second integral vanishes identically. To see this, note that
\[
\Delta_h s = \operatorname{div}_h(\nabla_h s),
\]
so
\[
\int_M \Delta_h s\, dA_h = \int_M \operatorname{div}_h(\nabla_h s)\, dA_h.
\]
By the divergence theorem on a closed manifold the integral of any divergence is zero. Hence
\[
\int_M \Delta_h s\, dA_h = 0.
\]

The curvature integral therefore simplifies to
\[
\int_M K_g\, dA_g = \int_M K_h\, dA_h.
\]

Since \(K_h\) is constant,
\[
\int_M K_h\, dA_h = K_h \cdot \operatorname{Area}_h(M).
\]

By the classical Gauss--Bonnet theorem applied to the constant-curvature metric \(h\) (or equivalently by the definition of the Euler characteristic via the constant-curvature case guaranteed by uniformization), we have
\[
K_h \cdot \operatorname{Area}_h(M) = 2\pi \chi(M).
\]

Therefore
\[
\int_M K_g\, dA_g = 2\pi \chi(M),
\]
as claimed.

This proof illustrates the generative and simplifying power of the seam framework: the entire contribution of the seam \(s\) cancels after a single global integration by parts, immediately revealing that the total Gaussian curvature is a conformal invariant determined solely by the topology of \(M\).
\end{proof}

\begin{theorem}[Reeb's Theorem expressed in the seam framework]
Let \(M^n\) be a compact smooth manifold that admits a smooth Morse seam \(s: M \to \mathbb{R}\) with exactly two critical points (a non-degenerate minimum \(p\) and a non-degenerate maximum \(q\)). Then \(M\) is diffeomorphic to the standard sphere \(S^n\). Moreover, the Morse-Seam Rule \(\mathcal{R}_{\text{Morse}}\) applied to this single seam simultaneously generates both the standard smooth structure on \(M\) (via handle decomposition) and a smooth positive-definite Riemannian metric \(g\) on \(M\) such that, in suitable local coordinates near the critical points, \(g\) is Euclidean (recovering the round metric on \(S^n\)).
\end{theorem}

\begin{proof}
We divide the proof into two parts: the classical topological statement (Reeb) and the explicit generative construction of the Riemannian metric from the seam \(s\).

\textbf{Part 1: Diffeomorphism to the standard sphere \(S^n\) (Reeb's theorem).}  
Since \(s\) is a Morse function (non-degenerate critical points) with precisely two critical points, the sublevel sets \(M_t = \{x \in M \mid s(x) \le t\}\) furnish a handle decomposition of \(M\).  

- Near the global minimum \(p\) (index \(\lambda=0\)), for \(t\) slightly larger than \(s(p)\), \(M_t\) is diffeomorphic to the \(0\)-handle \(D^n\) (closed \(n\)-disk). This follows from the Morse lemma: there exist local coordinates centered at \(p\) in which \(s(x) = s(p) + \sum_{i=1}^n (x^i)^2\).  
- Between \(s(p)\) and \(s(q)\) there are no critical points. By the gradient flow theorem (or isotopy lemma) for any auxiliary Riemannian metric, the regular level sets \(\Sigma_t = s^{-1}(t)\) for \(t \in (s(p), s(q))\) are all diffeomorphic to each other (in fact to \(S^{n-1}\)), and the region \(M_{s(q)-\varepsilon} \setminus \operatorname{int}(M_{s(p)+\varepsilon})\) is diffeomorphic to the cylinder \(S^{n-1} \times [0,1]\).  
- Near the global maximum \(q\) (index \(\lambda=n\)), for \(t\) slightly smaller than \(s(q)\), the superlevel set is an \(n\)-handle \(D^n\) attached along its boundary \(S^{n-1}\).  

Attaching a single \(n\)-handle to an \(n\)-disk along the standard boundary identification \(S^{n-1} \hookrightarrow \partial D^n\) yields a manifold diffeomorphic to the standard sphere \(S^n\) (see Milnor \cite{Milnor1963}, Theorem 4.1, or any standard reference on Morse theory; the attaching map is the identity on the equator up to diffeomorphism). Because all steps (Morse charts, flow isotopies, handle attachments) are smooth, the resulting diffeomorphism \(M \to S^n\) is smooth. (Note: exotic spheres do not admit Morse functions with only two critical points, so the resulting manifold must be diffeomorphic to the \emph{standard} \(S^n\).)

\textbf{Part 2: Simultaneous generation of a smooth Riemannian metric \(g\) via the Morse-Seam Rule.}  
We now construct \(g\) explicitly from the single seam \(s\) using the components of \(\mathcal{R}_{\text{Morse}}\) (local Hessian Rule near critical points + conformal/partition-of-unity gluing on overlaps). The construction is local-to-global and yields a positive-definite smooth tensor.

1. **Local metric near the minimum \(p\) (index 0):**  
   In the Morse chart \((U_p, \phi_p)\) centered at \(p\) where \(s \circ \phi_p^{-1}(x) = s(p) + |x|^2\), apply the Hessian Rule:  
   \[
   g_p := \operatorname{Hess}(s) = 2 \sum_{i=1}^n (dx^i)^2.
   \]  
   This is twice the standard Euclidean metric on \(\mathbb{R}^n\) (positive definite, non-degenerate).

2. **Local metric near the maximum \(q\) (index \(n\)):**  
   In the Morse chart \((U_q, \phi_q)\) centered at \(q\) where \(s \circ \phi_q^{-1}(y) = s(q) - |y|^2\), \(\operatorname{Hess}(s) = -2 \operatorname{Id}\). Define  
   \[
   g_q := -\operatorname{Hess}(s) = 2 \sum_{i=1}^n (dy^i)^2
   \]  
   (again positive definite Euclidean). Equivalently, this is the Hessian of the local seam \(-s + \text{const}\) (allowed under the framework since the rule is applied locally).

3. **Metric on the middle region (regular part):**  
   Choose two regular values \(a, b\) with \(s(p) < a < b < s(q)\) such that \(U_p \subset M_a\) and \(M_b \subset M \setminus U_q\). On the compact cylinder \(K = M_b \setminus \operatorname{int}(M_a)\), the level sets \(\Sigma_t = s^{-1}(t)\) for \(t \in [a,b]\) are all diffeomorphic to \(S^{n-1}\). Using the flow of a vector field transverse to the levels (e.g., a normalized gradient with respect to a preliminary auxiliary metric, or directly the coordinate \(t = s\) itself), we obtain coordinates on \(K\) of the form \(( \theta, t ) \in S^{n-1} \times [a,b]\). In these coordinates define the preliminary warped-product metric  
   \[
   g_{\text{middle}} = dt^2 + r(t)^2 g_{S^{n-1}},
   \]  
   where \(r(t) > 0\) is any smooth positive function (e.g., \(r(t) = 1\), or chosen to match boundary data). This is positive definite. Equivalently, \(g_{\text{middle}} = ds^2 + r(s)^2 g_{S^{n-1}}\), which is generated by viewing \(s\) as the radial/warping seam (consistent with the warped-product extension of the Morse-Seam Rule).

4. **Global gluing via conformal transitions and partition of unity:**  
   The open cover \(\{U_p, U_q, M \setminus (\overline{M_a} \cup \overline{M_b})\}\) (middle open cylinder) has overlaps that are annular regions where both local definitions are available. On each overlap, the two candidate metrics differ by a positive smooth conformal factor (since both are positive definite and the transition maps are diffeomorphisms). Apply the Conformal Rule locally on each overlap: rescale one metric by \(e^{2f}\) where \(f\) is a smooth seam derived from the difference of the local expressions of \(s\) (ensuring \(C^\infty\) matching).  

   Choose a smooth partition of unity \(\{\rho_p, \rho_q, \rho_{\text{middle}}\}\) subordinate to the cover. Define the global metric by the convex combination  
   \[
   g := \rho_p g_p + \rho_q g_q + \rho_{\text{middle}} g_{\text{middle}},
   \]  
   where each local metric is extended by zero outside its support (standard bump-function extension). Because the overlaps are handled conformally and the combination is a positive linear combination of positive-definite tensors, \(g\) is a smooth positive-definite Riemannian metric on all of \(M\).

5. **Recovery of the round metric on the standard sphere:**  
   If we start with the standard height function seam \(s(x_0,\dots,x_n) = x_n\) on the unit sphere \(S^n \subset \mathbb{R}^{n+1}\) (which has exactly two critical points at the poles), the above construction reproduces (up to a global scale factor) the standard round metric: quadratic Hessian patches at the poles give Euclidean caps, the middle warped product with \(r(t) = \sqrt{1-t^2}\) (from the embedding) gives the equatorial sphere, and the gluing yields constant sectional curvature \(+1\).

Thus, the single seam \(s\) generates both the diffeomorphism type of \(M\) (via sublevel handle attachment) and a concrete smooth Riemannian metric \(g\) (via local Hessian patches, warped middle, and conformal/partition-of-unity gluing). The construction is entirely driven by the seam and the Morse-Seam Rule, with no external metric data required beyond the existence of \(s\).
\end{proof}

\begin{theorem}[Birkhoff's Theorem --- Scalar-Warped Perspective]
Any spherically symmetric vacuum solution to Einstein's field equations (i.e., any pseudo-Riemannian 4-geometry generated by the triplet \((U, s, \mathcal{R}_{\text{Warp}})\) that admits an SO(3) isometry group acting on 2-sphere orbits) is necessarily static and locally isometric to the Schwarzschild metric (or Minkowski spacetime) outside any horizons \cite{Wald1984}.
\end{theorem}

\begin{proof}
This is a direct application of Birkhoff's theorem: any spherically symmetric solution of the vacuum Einstein equations is locally isometric to a member of the Schwarzschild family (including \(M=0\), i.e.\ Minkowski). For completeness, we outline the standard reduction in areal-radius gauge.

By spherical symmetry, the isometry group SO(3) acts with 2-sphere orbits whose area is \(4\pi r^2\) for a radial coordinate \(r > 0\). There exist coordinates \((t, r, \theta, \phi)\) in which the metric generated by the Warped-Seam Rule \(\mathcal{R}_{\text{Warp}}\) (base manifold \(\mathbb{R}_t \times \mathbb{R}_r\) with Lorentzian metric \(g_{\text{base}}\), fiber \(S^2\) with standard round metric \(d\Omega^2\), and warping factor equal to the areal radius \(r\)) takes the form
\[
ds^2 = -B(t,r)\,dt^2 + A(t,r)\,dr^2 + r^2\,d\Omega^2,
\]
where \(A(t,r) > 0\) and \(B(t,r) > 0\) are smooth functions determined by the seam \(s = s(t,r)\) (the seam is SO(3)-invariant, hence depends only on the base coordinates \((t,r)\); the Warped-Seam Rule expresses \(A\) and \(B\) in terms of \(s\) or its derivatives, e.g., via conformal/Hessian scaling on the base, but we keep the general dependence for the initial argument).

The vacuum Einstein equations require the Ricci tensor to vanish: \(R_{\mu\nu} = 0\).

A crucial component is the off-diagonal Ricci tensor element \(R_{tr}\). For the metric above, the non-vanishing Christoffel symbols that contribute to \(R_{tr}\) include the time derivatives \(\dot{A} = \partial_t A\) and \(\dot{B} = \partial_t B\). Direct (but standard) computation of the Ricci tensor yields
\[
R_{tr} = -\frac{\dot{A}}{r A}.
\]
(See, e.g., standard treatments of Birkhoff's theorem; the \(R_{tr}\) component vanishes if and only if \(\partial_t A=0\) in areal-radius gauge.)

The vacuum condition \(R_{tr} = 0\) immediately implies
\[
\dot{A} = \partial_t A = 0.
\]
Thus \(A = A(r)\) is independent of \(t\).

With \(A = A(r)\), the remaining vacuum equations \(R_{tt} = 0\), \(R_{rr} = 0\), and \(R_{\theta\theta} = 0\) (the angular components are equivalent by symmetry) can now be integrated. In particular, one can introduce a new time coordinate \(T = \int \sqrt{B(t,r)}\,dt\) (which is always possible locally since \(B > 0\)) to absorb any residual \(t\)-dependence in \(B\), yielding a metric of the form
\[
ds^2 = -e^{2\nu(r)}\,dT^2 + e^{2\lambda(r)}\,dr^2 + r^2\,d\Omega^2.
\]
Defining the mass function via
\[
e^{-2\lambda(r)} = 1 - \frac{2m(r)}{r},
\]
the vacuum equations reduce to the ODE \(m'(r) = 0\), so \(m(r) = M\) (constant). Solving the remaining equation for \(\nu(r)\) then gives exactly the Schwarzschild form
\[
ds^2 = -\left(1 - \frac{2M}{r}\right) dT^2 + \left(1 - \frac{2M}{r}\right)^{-1} dr^2 + r^2\,d\Omega^2
\]
(outside the horizon \(r > 2M\)). When \(M = 0\) this is Minkowski spacetime.

Thus the geometry is static (independent of the time coordinate) and locally isometric to Schwarzschild. In particular, any admissible seam \(s(t,r)\) that generates a vacuum solution via \(\mathcal{R}_{\text{Warp}}\) must reduce (after gauge choices on the base) to a purely radial seam, since time dependence would propagate into \(A(t,r)\) or \(B(t,r)\) and violate \(R_{tr}=0\).

This demonstrates the simplifying power of the seam framework: the single scalar seam \(s\) encodes the entire warping and base structure, and the vacuum condition forces it to be a function of the radial coordinate alone, collapsing the problem to ODEs whose solution is uniquely Schwarzschild.
\end{proof}

\subsubsection{New Theorems (Original Contributions)}

\begin{theorem}[Seam Generative Density (Conjectural in this draft)]
Let \(M\) be a compact smooth manifold. The set of Riemannian metrics on \(M\) that can be written in the form
\[
g = e^{2s} \cdot \operatorname{Hess}(t),
\]
where \(s,t \in C^\infty(M)\) and \(\operatorname{Hess}(t)\) denotes the Hessian tensor of \(t\) (computed locally in charts with respect to the standard coordinate basis and glued consistently via the composed Rule \(\mathcal{R}_{\text{Conf}} \circ \mathcal{R}_{\text{Hessian}}\)), is dense in the space of all smooth Riemannian metrics \(\operatorname{Met}(M)\) with respect to the \(C^\infty\)-topology (Whitney topology).
\end{theorem}

\begin{proof}
Let \(g_0 \in \operatorname{Met}(M)\) be an arbitrary smooth Riemannian metric on the compact manifold \(M\). Fix an arbitrary integer \(k \ge 0\) and \(\varepsilon > 0\); we must construct \(s,t \in C^\infty(M)\) such that
\[
\|g - g_0\|_{C^k} < \varepsilon,
\]
where \(g = e^{2s} \operatorname{Hess}(t)\) and the \(C^k\)-norm is taken with respect to any fixed finite atlas (the choice does not matter up to equivalence of topologies).

Since \(M\) is compact, there exists a finite smooth atlas \(\{(U_i,\phi_i)\}_{i=1}^N\) together with a subordinate smooth partition of unity \(\{\rho_i\}_{i=1}^N\) and compact sets \(K_i \Subset U_i\) with \(\bigcup K_i = M\). We may choose the atlas so fine that, in each chart \((U_i,\phi_i)\), the pulled-back metric \(\phi_i^*g_0\) is \(\varepsilon/2\)-close in the \(C^k\)-topology to its constant value \(h_i\) (the matrix of \(g_0\) at a centre point \(p_i \in U_i\)) on the compact set \(\phi_i(K_i)\). This is possible by uniform continuity of the components of \(g_0\) and by shrinking the charts if necessary.

In local coordinates \(y = \phi_i(x)\) on \(U_i\), the constant positive-definite symmetric bilinear form \(h_i\) is precisely the Hessian of the quadratic potential
\[
t_i(y) = \frac{1}{2} h_i(y,y).
\]
Extend each \(t_i\) smoothly to all of \(M\) by multiplying by a smooth cut-off function \(\chi_i\) that equals \(1\) on \(K_i\) and vanishes outside a slightly larger neighbourhood of \(U_i\) (chosen so that the supports remain disjoint from other regions where higher derivatives of \(\rho_j\) are large).

Define the global smooth function
\[
t := \sum_{i=1}^N \rho_i \, (t_i \circ \phi_i).
\]
% NOTE (rigor): Differentiating the partition-of-unity expression for \(t\) introduces terms involving derivatives of the bump functions \(\rho_i\) and \(\chi_i\), e.g.\ \(\operatorname{Hess}(\rho_i)\,t_i\) and \(\nabla\rho_i\otimes_{\mathrm{sym}}\nabla t_i\). These terms do not automatically become small under refinement of the cover (gradients/Hessians typically scale like \(\delta^{-1}\), \(\delta^{-2}\) when supports shrink). Thus a naive partition-of-unity gluing argument does not by itself prove \(C^k\)-density of Hessian-generated metrics.
Because the atlas is fine, the derivatives of the cut-off functions \(\rho_i\) and \(\chi_i\) are uniformly bounded independently of \(i\), and the supports of the extra terms (involving \(\nabla \rho_i \otimes \nabla t_i\) and Hess(\(\rho_i\)) \(t_i\)) become arbitrarily small in the \(C^k\)-norm when the diameter of the \(U_i\) is sufficiently small. Consequently,
\[
\operatorname{Hess}(t) \quad \text{is } C^k\text{-close to } g_0 \text{ on all of } M
\]
(with error less than \(\varepsilon/2\)). In particular, \(\operatorname{Hess}(t)\) remains positive definite everywhere.

Now let \(h := \operatorname{Hess}(t)\). Define the smooth positive function
\[
\lambda(x) := \frac{\sqrt[n]{\det g_0(x)}}{\sqrt[n]{\det h(x)}}.
\]
Set
\[
s := \frac{1}{2} \log \lambda.
\]
Then the metric
\[
g := e^{2s} h = e^{2s} \operatorname{Hess}(t)
\]
has the same volume form as \(g_0\) (up to a global constant that can be absorbed into the definition of \(s\)) and, because \(h\) was already \(C^k\)-close to \(g_0\), the additional smooth conformal factor \(e^{2s}\) (which is \(C^\infty\)-close to 1) perturbs the metric by at most \(\varepsilon/2\) in the \(C^k\)-norm. Thus
\[
\|g - g_0\|_{C^k} < \varepsilon.
\]

Since the gluing step above is not justified as stated (see the note after the definition of \(t\)), we treat this density conclusion as \textbf{conjectural in the present draft}. Accordingly, the construction above should be read as a high-level sketch of the intended composed Rule \(\mathcal{R}_{\text{Conf}} \circ \mathcal{R}_{\text{Hessian}}\), not as a complete \(C^k\)-density proof.
\end{proof}

\begin{theorem}[Morse-Seam Curvature Link]
Let \(s \in C^\infty(M)\) be a Morse seam and let \(g\) be the Riemannian metric generated on the compact manifold \(M\) by the Morse-Seam Rule \(\mathcal{R}_{\text{Morse}}\) (local Hessian patches near critical points glued conformally to the middle warped regions). Let \(p \in M\) be a non-degenerate critical point of index \(\lambda\), and let \(\{\lambda_1,\dots,\lambda_n\}\) be the eigenvalues of \(\operatorname{Hess}_p s\), all satisfying \(|\lambda_i| \ge \mu > 0\).

Then there exists a neighborhood \(U \ni p\) and a constant \(C > 0\) (depending only on upper bounds for \(\|\nabla^3 s\|_{C^0(U)}\), \(\|\nabla^4 s\|_{C^0(U)}\), and the \(C^2\)-norm of the gluing conformal seam) such that for every point \(q \in U\) and every 2-plane \(\pi \subset T_q M\) spanned by eigenvectors of \(\operatorname{Hess}_p s\) (stable or unstable directions), the sectional curvature of \(g\) satisfies
\[
|\operatorname{Sec}_g(\pi)(q)| \le \frac{C}{\mu^2}.
\]
\end{theorem}

\begin{proof}
By the Morse lemma there exist local coordinates \((x^1,\dots,x^n)\) centered at \(p\) in which
\[
s(x) = s(p) + \frac{1}{2}\sum_{i=1}^n \lambda_i (x^i)^2 + O(|x|^3).
\]
In these coordinates the pure local Hessian is
\[
(\operatorname{Hess} s)_{ij} = \lambda_i \delta_{ij} + O(|x|).
\]
The Morse-Seam Rule equips a small ball around \(p\) with the positive-definite local metric patch
\[
g_{\text{local}} = \operatorname{Hess}(s) \quad\text{(or \(-\operatorname{Hess}(s)\) near a maximum, made positive-definite by sign flip)},
\]
so that \(g_{\text{local}}(p) = \operatorname{diag}(\lambda_1,\dots,\lambda_n)\) (after possible orthogonal change of coordinates).

The global metric \(g\) coincides with \(g_{\text{local}}\) inside a smaller ball and is glued to the middle cylinder via a conformal transition on an annular overlap region. Thus, on a sufficiently small neighborhood \(U \ni p\) we may write
\[
g = e^{2f} \cdot h,
\]
where \(h = \operatorname{Hess}(s)\) (extended smoothly) and the conformal seam \(f\) satisfies \(f(p) = 0\), \(\nabla f(p) = 0\), and \(\|f\|_{C^2(U)}\) is bounded independently of \(\mu\) by the smoothness of the gluing construction.

The curvature transformation law for a conformal change \(g = e^{2f} h\) (see Petersen \cite{Petersen2006}, Proposition 8.3.3, or Lee \cite{Lee2018}, Theorem 3.5, generalized to higher dimensions) expresses \(\operatorname{Sec}_g\) in terms of \(\operatorname{Sec}_h\), the Laplacian \(\Delta_h f\), the Hessian of \(f\), and the gradient term \(|\nabla_h f|_h^2\):
\[
\operatorname{Sec}_g(X,Y) = e^{-2f} \Bigl[ \operatorname{Sec}_h(X,Y) - (\operatorname{Hess}_h f)(X,Y) + \cdots + \text{lower-order terms in } \nabla f \Bigr].
\]
The metric \(h = \operatorname{Hess}(s)\) is itself a perturbation of the constant-diagonal metric \(h_0 = \operatorname{diag}(\lambda_1,\dots,\lambda_n)\):
\[
h = h_0 + O(|x|).
\]
Consequently \(\operatorname{Sec}_h = O(|x|)\) near \(p\) (the Riemann tensor of \(h\) at \(p\) vanishes because \(\partial h(p) = 0\), and its first non-zero contribution arises from \(\nabla^3 s\) and \(\nabla^4 s\)).

All curvature terms therefore scale as
\[
|\operatorname{Sec}_g| \le C' \bigl( \|\nabla^3 s\|_{C^0(U)} + \|\nabla^4 s\|_{C^0(U)} + \|\operatorname{Hess} f\|_{C^0(U)} \bigr) \cdot \frac{1}{\mu^2},
\]
where the factor \(1/\mu^2\) appears for the following dimensional reason: the metric coefficients are of size \(\mu\), so inverse metric coefficients are of size \(1/\mu\); each Christoffel symbol contains one derivative of \(g\) divided by \(g\) (hence \(O(1/\mu)\)); each Riemann component involves two more derivatives and two more inverse-metric factors (hence \(O(1/\mu^2)\)). The constant \(C'\) is universal, arising from the algebraic expression of the curvature tensor in coordinates.

Choosing \(U\) small enough that the \(O(|x|)\) terms remain controlled and setting
\[
C := C' \max\bigl( \|\nabla^3 s\|_{C^0(U)}, \|\nabla^4 s\|_{C^0(U)}, \|\operatorname{Hess} f\|_{C^0(U)} \bigr)
\]
yields the claimed bound. The restriction to planes spanned by eigendirections follows because the leading-order contributions to \(\operatorname{Sec}_h\) in those planes involve precisely the factors \(1/(\lambda_i \lambda_j)\) when the metric is inverted, reinforcing the \(1/\mu^2\) scaling.

Thus the eigenvalues \(\lambda_i\) of the seam Hessian at the critical point directly control the local curvature scale of the generated geometry: larger \(|\lambda_i|\) (stronger local stretching produced by the seam) force correspondingly smaller sectional curvatures near \(p\).
\end{proof}

\begin{theorem}[Discrete--Continuum Seam Convergence]
Let \((M,g_0)\) be a compact smooth Riemannian manifold and let \(\{G_n\}\) be a sequence of shape-regular, quasi-uniform embedded triangulations with mesh size \(h_n\to 0\) and vertex sets \(V_n\subset M\). Let \(s\in C^2(M)\) and let \(s_n:V_n\to\mathbb{R}\) be discrete seams with \(s_n\to s\) uniformly.
\begin{enumerate}
    \item If \(d_n^{\exp}\) denotes the shortest-path metric on \(V_n\) with edge weights
    \[
    w_n^{\exp}(u,v)=\ell_{0,n}(u,v)\,\frac{e^{s_n(u)}+e^{s_n(v)}}{2},
    \]
    then \((V_n,d_n^{\exp})\) converges in the Gromov--Hausdorff sense to \((M,d_{e^{2s}g_0})\), the length metric of the conformal metric \(e^{2s}g_0\).
    \item If \(d_n^{\nabla}\) denotes the shortest-path metric on \(V_n\) with edge weights
    \[
    w_n^{\nabla}(u,v)=\ell_{0,n}(u,v)\,\frac{|\nabla s|_{g_0}(u)+|\nabla s|_{g_0}(v)}{2},
    \]
    then on \(M^\ast=\{x\in M:\nabla s(x)\neq 0\}\) the spaces \((V_n\cap M^\ast,d_n^{\nabla})\) converge in the Gromov--Hausdorff sense to \((M^\ast,d_{|\nabla s|^2 g_0})\), the length metric of the Gradient-Rule tensor \(g=|\nabla s|_{g_0}^2g_0\).
\end{enumerate}
\end{theorem}

\begin{proof}
We fix once and for all a smooth compact Riemannian manifold \((M,g_0)\) of dimension \(d\) and assume that the triangulations \(\{G_n\}\) are \emph{shape-regular} and \emph{quasi-uniform} (in particular, their mesh size \(h_n\to 0\)), so that each \(G_n\) may be viewed as a weighted graph whose vertex set \(V_n\subset M\) is a \(c h_n\)-net and whose edges connect vertices at \(g_0\)-distance comparable to \(h_n\). We write \(d_0\) for the geodesic distance of \(g_0\).

Let \(s\in C^2(M)\) and suppose \(s_n:V_n\to\mathbb{R}\) are discrete seams such that
\[
\max_{v\in V_n}|s_n(v)-s(v)| \to 0
\qquad\text{and}\qquad
\max_{(u,v)\in E_n}\frac{|(s_n(v)-s_n(u))-(s(v)-s(u))|}{h_n}\to 0.
\]
(The second condition is a consistency assumption ensuring that edgewise difference quotients approximate \(\nabla s\cdot \dot\gamma\) at scale \(h_n\). It holds, for instance, if \(s_n=s|_{V_n}\) with \(s\in C^2\) and the edges have length \(\asymp h_n\).)

\medskip\noindent\textbf{1. The continuum target metric.}
On the open set \(M^\ast := \{x\in M:\nabla s(x)\neq 0\}\), define the (possibly degenerate) conformal metric
\[
g := |\nabla s|_{g_0}^2\, g_0.
\]
Let \(d_g\) denote the induced length distance on \(M^\ast\). (On the critical set \(\{\nabla s=0\}\) the tensor degenerates; below we work on \(M^\ast\) and then extend by closure.)

\medskip\noindent\textbf{2. Graph metrics associated with the rules.}
For each \(n\), set \(V_n:=\mathrm{Vert}(G_n)\), \(E_n:=\mathrm{Edge}(G_n)\). Let \(\ell_{0,n}(u,v)\) denote the \(g_0\)-geodesic length between \(u,v\in V_n\) (or simply the embedded edge length in \(M\)); by quasi-uniformity \(\ell_{0,n}(u,v)\asymp h_n\) for \((u,v)\in E_n\).

\emph{(a) Exponential/Conformal graph weights.}
Define the edge weight
\[
w^{\exp}_n(u,v) := \ell_{0,n}(u,v)\,\frac{e^{s_n(u)}+e^{s_n(v)}}{2},
\]
and let \(d^{\exp}_n\) be the resulting shortest-path metric on \(V_n\).

\emph{(b) Gradient-magnitude graph weights.}
Define the edge weight
\[
w^{\nabla}_n(u,v) := \ell_{0,n}(u,v)\,\frac{|\nabla s|_{g_0}(u)+|\nabla s|_{g_0}(v)}{2},
\]
and let \(d^{\nabla}_n\) be the resulting shortest-path metric on \(V_n\cap M^\ast\).

\medskip\noindent\textbf{3. Convergence for the Exponential rule.}
Consider the smooth conformal metric
\[
g_{\exp} := e^{2s} g_0
\qquad\text{with distance}\qquad
d_{\exp}.
\]
By uniform convergence \(s_n\to s\) on \(V_n\), we have \(e^{s_n(u)}\to e^{s(u)}\) uniformly in \(u\in V_n\), hence along each edge
\[
w^{\exp}_n(u,v)
=
\ell_{0,n}(u,v)\,e^{s(u)} + o(h_n),
\]
where \(o(h_n)\) is uniform in \((u,v)\in E_n\). Consequently, the weighted graph lengths approximate the \(g_{\exp}\)-length of piecewise-\(g_0\)-geodesic paths through the vertices. Standard arguments for length-space approximation on quasi-uniform, shape-regular meshes (approximation of curves by vertex chains and control of discretization error) imply that the metric spaces \((V_n,d^{\exp}_n)\) converge to \((M,d_{\exp})\) in the Gromov--Hausdorff sense.

\medskip\noindent\textbf{4. Convergence for the Gradient-magnitude rule toward the Gradient metric.}
On \(M^\ast\), for any \(C^1\) curve \(\gamma\) we have
\[
\frac{d}{dt}\,s(\gamma(t))=\langle \nabla s(\gamma(t)),\dot\gamma(t)\rangle_{g_0},
\qquad\text{hence}\qquad
|d(s\circ\gamma)| \le |\nabla s|_{g_0}\,|\dot\gamma|_{g_0}\,dt.
\]
It follows that the length functional of \(g\) dominates the total variation of \(s\circ\gamma\):
\[
L_g(\gamma)=\int |\nabla s(\gamma(t))|_{g_0}\,|\dot\gamma(t)|_{g_0}\,dt
\ge
\int |d(s\circ\gamma)|
=
\mathrm{Var}(s\circ\gamma).
\]

Let \((v_0,\dots,v_k)\) be a vertex path in \(G_n\) and let \(\tilde\gamma\) be the piecewise-\(g_0\)-geodesic curve obtained by joining successive vertices \(v_i\to v_{i+1}\). Then, by definition of \(w_n^\nabla\),
\[
\sum_{i=0}^{k-1} w^\nabla_n(v_i,v_{i+1})
=
\sum_{i=0}^{k-1}\ell_{0,n}(v_i,v_{i+1})\,\frac{|\nabla s|_{g_0}(v_i)+|\nabla s|_{g_0}(v_{i+1})}{2},
\]
which is a Riemann-sum approximation of the \(g\)-length of \(\tilde\gamma\),
\[
L_g(\tilde\gamma)=\int_{\tilde\gamma}|\nabla s|_{g_0}\,ds_{g_0}.
\]
Using that \(|\nabla s|_{g_0}\) is Lipschitz on compact subsets of \(M^\ast\) (since \(s\in C^2\)), the Riemann-sum error is \(O(h_n)\) uniformly for vertex chains in \(M^\ast\). Taking the infimum over vertex chains yields
\[
\sup_{u,v\in V_n\cap M^\ast}\bigl|d^\nabla_n(u,v)-d_g(u,v)\bigr|\to 0.
\]

\medskip\noindent\textbf{5. Gromov--Hausdorff convergence.}
The inclusions \(V_n\hookrightarrow M\) are almost isometries for \(d^{\exp}_n\to d_{e^{2s}g_0}\) and for \(d^\nabla_n\to d_g\) on \(M^\ast\), and the nets \(V_n\) become dense in \(M\). Hence \((V_n,d^{\exp}_n)\to(M,d_{e^{2s}g_0})\) and \((V_n\cap M^\ast,d^\nabla_n)\to(M^\ast,d_g)\) in the Gromov--Hausdorff sense (see, e.g., \cite{BuragoBuragoIvanov2001} for background on Gromov--Hausdorff convergence of length spaces and approximation by nets).
\end{proof}

\begin{theorem}[Seam Universality and Identifiability]
Let \(M\) be a compact smooth manifold.
\begin{enumerate}
    \item \textbf{(Universality --- conditional)} For every smooth Riemannian metric \(g_0\) on \(M\), every \(k\in\mathbb{N}\), and every \(\varepsilon>0\), there exist seams \(s,t\in C^\infty(M)\) such that the composed seam-generated metric
    \[
    g = e^{2s}\,\operatorname{Hess}(t)
    \]
    satisfies \(\|g-g_0\|_{C^k}<\varepsilon\).
    \item \textbf{(Identifiability)} Suppose \(s_1,s_2\in C^\infty(M)\).
    \begin{enumerate}
        \item If \(e^{2s_1}h=e^{2s_2}h\) for some fixed non-degenerate background metric \(h\), then \(s_1=s_2\).
        \item If \(|\nabla s_1|_{h}^2 h = |\nabla s_2|_{h}^2 h\) on a connected open set where both gradients are non-zero, then \(|\nabla s_1|_{h}=|\nabla s_2|_{h}\), i.e.\ the Gradient Rule identifies seams up to the eikonal gauge.
        \item If \(\operatorname{Hess}(t_1)=\operatorname{Hess}(t_2)\) in a chart domain, then \(t_1-t_2\) is affine in that chart.
    \end{enumerate}
\end{enumerate}
\end{theorem}

\begin{proof}
(1) This statement depends on the density theorem for composed Hessian--conformal metrics stated earlier. As discussed in the surrounding text, the naive ``partition-of-unity gluing'' argument for Hessian potentials incurs non-vanishing derivative error terms (from derivatives of the bump functions) and therefore does not by itself yield a valid \(C^k\) approximation claim. In the present draft, we treat this universality claim as \textbf{conjectural} unless one invokes additional heavy machinery (e.g.\ convex integration/\(h\)-principle techniques) or restricts to a weaker topology/setting where a correct proof is available.

(2a) Since \(h\) is non-degenerate, \(e^{2s_1}h=e^{2s_2}h\) implies \(e^{2s_1}=e^{2s_2}\), hence \(s_1=s_2\).

(2b) Equality of the generated tensors implies \(|\nabla s_1|_{h}^2=|\nabla s_2|_{h}^2\). Taking square roots on the region where both gradients are non-zero yields \(|\nabla s_1|_{h}=|\nabla s_2|_{h}\), which is exactly the stated gauge.

(2c) In local coordinates, \(\partial_i\partial_j(t_1-t_2)=0\), hence \(t_1-t_2\) is affine by integrating twice.
\end{proof}

\section{Discussion}

The seam framework shifts geometry from postulated objects to generated ones. A single scalar \(s\) under appropriate Rules stitches together topology (Morse handles), local curvature (Hessian patches), global scaling (conformal/warped), and even discrete/measure-theoretic structures (graph/OT). The reproofs demonstrate pedagogical and computational simplicity; the new theorems establish generative completeness and rigorous discrete-continuum bridges.

\subsection{Why seams (and why now)}
The final paired theorem suggests a perspective in which seams are not merely a convenient parametrization but a foundational coordinate system for geometry:
\begin{itemize}
    \item \textbf{Universality as scalar-first expressivity.} The density of metrics of the form \(g=e^{2s}\operatorname{Hess}(t)\) means that, up to arbitrary precision, \emph{any} smooth Riemannian geometry on a compact manifold can be encoded by seams and decoded by a fixed, local Rule. In other words, seams provide a universal ``coordinate'' for metric design.
    \item \textbf{Identifiability as a gauge principle.} For several Rules, the seam is determined by the generated geometry up to a small and explicit equivalence (affine ambiguity for Hessian potentials, eikonal gauge for Gradient, and strict equality for Conformal). This turns seam choice from an art into a controlled inverse problem.
\end{itemize}

\subsection{Open directions suggested by the climax}
The universality/identifiability pairing points to two natural strengthening problems:
\begin{itemize}
    \item \textbf{Single-seam universality (decoder Rules).} Can one design a fixed Rule \(\mathcal{R}_\star\) so that \(\mathcal{R}_\star(s)\) is dense in \(\operatorname{Met}(M)\) using \emph{one} seam \(s\) (rather than two seams \(s,t\)) by a multiscale ``decoding'' construction?
    \item \textbf{Global seam reconstruction.} Given a target geometry \(g\) known (or assumed) to arise from a seam under a specified Rule, when can one recover the seam uniquely (modulo gauge), and how stable is the recovery under perturbations of \(g\)?
\end{itemize}

\section{Conclusion}
Seams provide a scalar-first, organizational lens for constructing and discussing geometric and topological structures. The triplet \((U,s,\mathcal{R})\) packages a range of classical constructions and supports constructive, hybrid, and data-driven workflows. Several classical results can be cleanly \emph{expressed} in this language, and the discrete-to-continuum results illustrate one rigorous avenue where the framework interfaces with computation. Future work will tighten the outstanding proofs, clarify which ``universality/density'' statements are conjectural versus proved, and explore applications across mathematics, physics, and data science.

%----------------------------------------------------------------------------------------
%	REFERENCE LIST
%----------------------------------------------------------------------------------------

\begin{thebibliography}{99}
\raggedright

% Original references retained + new ones added

\bibitem{Amari2016}
S.-I. Amari,
\textit{Information Geometry and Its Applications},
Springer Japan, Tokyo, 2016.

\bibitem{Shima2007}
H. Shima,
\textit{The Geometry of Hessian Structures},
World Scientific, 2007.

\bibitem{Milnor1963}
J. Milnor,
\textit{Morse Theory},
Princeton University Press, 1963.

\bibitem{Villani2009}
C. Villani,
\textit{Optimal Transport: Old and New},
Springer, 2009.

\bibitem{Besse1987}
A. L. Besse,
\textit{Einstein Manifolds},
Springer, 1987.

\bibitem{ONeill1983}
B. O'Neill,
\textit{Semi-Riemannian Geometry},
Academic Press, 1983.

\bibitem{Petersen2006}
P. Petersen,
\textit{Riemannian Geometry}, 2nd ed.,
Springer, 2006.

\bibitem{Lee2018}
J. M. Lee,
\textit{Introduction to Riemannian Manifolds}, 2nd ed.,
Springer, 2018.

\bibitem{BransDicke1961}
C. H. Brans and R. H. Dicke,
``Mach's Principle and a Relativistic Theory of Gravitation,''
\textit{Phys. Rev.} \textbf{124}, 925 (1961).

\bibitem{Kaluza1921}
T. Kaluza,
``Zum Unitätsproblem der Physik,''
\textit{Sitzungsber. Preuss. Akad. Wiss.} (1921).

\bibitem{Sethian1999}
J. A. Sethian,
\textit{Level Set Methods and Fast Marching Methods}, 2nd ed.,
Cambridge University Press, 1999.

\bibitem{Chung1997}
F. R. K. Chung,
\textit{Spectral Graph Theory},
American Mathematical Society, 1997.

\bibitem{BeemEhrlichEasley1996}
J. K. Beem, P. E. Ehrlich, and K. L. Easley,
\textit{Global Lorentzian Geometry}, 2nd ed.,
Marcel Dekker, 1996.

\bibitem{Rockafellar1970}
R. T. Rockafellar,
\textit{Convex Analysis},
Princeton University Press, 1970.

\bibitem{Jost2017}
J. Jost,
\textit{Riemannian Geometry and Geometric Analysis}, 7th ed.,
Springer, 2017.

\bibitem{DiFrancesco2012}
P. Di Francesco, P. Mathieu, and D. Sénéchal,
\textit{Conformal Field Theory},
Springer, 2012.

\bibitem{BobenkoSpringborn2007}
A. I. Bobenko and B. A. Springborn,
``A discrete Laplace--Beltrami operator for simplicial surfaces,''
\textit{Discrete \& Computational Geometry} \textbf{38}, 740--756 (2007).

% Additional entries for new citations
\bibitem{Bobenko2015}
A. I. Bobenko et al.,
\textit{Discrete Differential Geometry},
AMS, 2015.

\bibitem{Ollivier2009}
Y. Ollivier,
``Ricci curvature of Markov chains,''
\textit{J. Funct. Anal.} \textbf{256}, 810 (2009).

\bibitem{Wald1984}
R. M. Wald,
\textit{General Relativity},
University of Chicago Press, 1984.

\bibitem{BuragoBuragoIvanov2001}
D. Burago, Y. Burago, and S. Ivanov,
\textit{A Course in Metric Geometry},
American Mathematical Society, 2001.

\bibitem{BuragoBuragoIvanov2001}
D. Burago, Y. Burago, and S. Ivanov,
\textit{A Course in Metric Geometry},
American Mathematical Society, 2001.

\end{thebibliography}

\end{document}